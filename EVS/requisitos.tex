\section{Requisitos}
\par A continución se va a proceder a explicar en detalle cada uno de los requisitos que se han identificado y que por tanto deberán ser satisfechos por el sistema. Para seguir un formato unificado para todos los requisitos, se ha decidido utilizar la siguiente plantilla:
\begin{table}[H]
\begin{center}
\begin{tabular}{p{3,5cm} p{7cm}}
\multicolumn{2}{c}{\textbf{Requisito: XX-YY} } \\
\hline \hline
\textbf{Nombre del Requisito} &   \\
\hline
\textbf{Descripción} &  \\
\hline
\textbf{Tipo} &   \\
\hline
\textbf{Fuente del Requisito} &   \\
\hline
\textbf{Prioridad} &   \\ \hline
\end{tabular}
\caption{Plantilla requisitos}
\label{tab:Plantilla-Requisitos}
\end{center}
\end{table}

\par En esta tabla se detallan los siguientes apartados:
\begin{itemize}[-]
\item \textbf{Requisito XX-YY: }
\par El codigo asignado a cada uno de los requisitos será una sigla correspondiente al tipo de requisito. Por lo tanto puede haber: RF (Requisito Funcional), RNF (Requisito No Funcional) y RI (Requisito de Interfaz). Las dos cifras siguientes corresponderan al numero de requisito dentro de cada uno de los apartados.
\item \textbf{Nombre del Requisito: }
\par Se especificará un nombre para el requisito que lo identifique unequivocamente.
\item \textbf{Descripción: }
\par Se realizará una descripción detallada y concisa del requisito en sí.
\item \textbf{Tipo: }
\par Los Requisitos Funcionales y los Requisitos de Interfaz se clasificarán como de tipo requisito, mientras que los Requisitos No Funcionales se clasificarán como de tipo restricción.
\item \textbf{Fuente del Requisito: }
\par En este punto se especificará la fuente del requisito, pudiendo ser del cliente o como el resultado del analisis del proyecto por parte del analista.
\item \textbf{Prioridad: }
\par La prioridad de un requisito se varía según sea un requisito cuya fuente ha sido el cliente y es una parte esencial que este ultimo comporobará, o bien, si es un requisito opcional que estaría bien su implementación pero no es necesario para el exito del proyecto.
\end{itemize}

\newpage

\subsection{Requisitos funcionales}


\subsubsection{Control de punto ciego}

\begin{table}[H]
\begin{center}
\begin{tabular}{p{3,5cm} p{7cm}}
\multicolumn{2}{c}{\textbf{Requisito: RF-01} } \\
\hline \hline
\textbf{Nombre del Requisito} & Detección punto ciego  \\
\hline
\textbf{Descripción} & El sistema tiene que detectar un objeto situado en la zona de punto ciego de los retrovisores exteriores del vehículo.  \\
\hline
\textbf{Tipo} & Requisito  \\
\hline
\textbf{Fuente del Requisito} & Cliente  \\
\hline
\textbf{Prioridad} & Alta/Esencial  \\ \hline
\end{tabular}
\caption{Requisito funcional 01}
\label{tab:RF-01}
\end{center}
\end{table}

\begin{table}[H]
\begin{center}
\begin{tabular}{p{3,5cm} p{7cm}}
\multicolumn{2}{c}{\textbf{Requisito: RF-02} } \\
\hline \hline
\textbf{Nombre del Requisito} & Notificar coche en punto ciego  \\
\hline
\textbf{Descripción} & El sistema tiene que avisar al conductor cuando un objeto se encuentre en la zona de punto ciego. \\
\hline
\textbf{Tipo} & Requisito  \\
\hline
\textbf{Fuente del Requisito} & Analista  \\
\hline
\textbf{Prioridad} & Alta/Esencial  \\ \hline
\end{tabular}
\caption{Requisito funcional 02}
\label{tab:RF-02}
\end{center}
\end{table}

\begin{table}[H]
\begin{center}
\begin{tabular}{p{3,5cm} p{7cm}}
\multicolumn{2}{c}{\textbf{Requisito: RF-03} } \\
\hline \hline
\textbf{Nombre del Requisito} & Notificación en el retrovisor \\
\hline
\textbf{Descripción} & El sistema avisará al conductor de que un objeto está en la zona de punto ciego mediante una notificación en el retrovisor exterior situado en el lateral en el que se encuentra dicho objeto.\\
\hline
\textbf{Tipo} & Requisito  \\
\hline
\textbf{Fuente del Requisito} & Analista  \\
\hline
\textbf{Prioridad} & Alta/Esencial  \\ \hline
\end{tabular}
\caption{Requisito funcional 03}
\label{tab:RF-03}
\end{center}
\end{table}

\begin{table}[H]
\begin{center}
\begin{tabular}{p{3,5cm} p{7cm}}
\multicolumn{2}{c}{\textbf{Requisito: RF-04} } \\
\hline \hline
\textbf{Nombre del Requisito} & Generación de notificación sonora \\
\hline
\textbf{Descripción} &  Cuando el conductor del vehículo active el intermitente indicando que se dispone a cambiar de carril y simultáneamente un objeto se encuentre en la zona de punto ciego del carril al que se pretende cambiar, el sistema emitirá una alerta sonora. \\
\hline
\textbf{Tipo} & Requisito  \\
\hline
\textbf{Fuente del Requisito} & Cliente  \\
\hline
\textbf{Prioridad} & Alta/Esencial  \\ \hline
\end{tabular}
\caption{Requisito funcional 04}
\label{tab:RF-04}
\end{center}
\end{table}


\subsubsection{Aviso de cambio de carril}

\begin{table}[H]
\begin{center}
\begin{tabular}{p{3,5cm} p{7cm}}
\multicolumn{2}{c}{\textbf{Requisito: RF-05} } \\
\hline \hline
\textbf{Nombre del Requisito} & Trayectoria del carril\\
\hline
\textbf{Descripción} & El sistema identificará la trayectoria del carril por el que el conductor se encuentra circulando. \\
\hline
\textbf{Tipo} & Requisito  \\
\hline
\textbf{Fuente del Requisito} & Analista \\
\hline
textbf{Prioridad} & Alta/Esencial  \\ \hline
\end{tabular}
\caption{Requisito funcional 05}
\label{tab:RF-05}
\end{center}
\end{table}

\begin{table}[H]
\begin{center}
\begin{tabular}{p{3,5cm} p{7cm}}
\multicolumn{2}{c}{\textbf{Requisito: RF-06} } \\
\hline \hline
\textbf{Nombre del Requisito} & Trayectoria del vehículo\\
\hline
\textbf{Descripción} & El sistema deberá detectar la trayectoria que va a seguir el vehículo en función de la velocidad y el ángulo del volante. \\
\hline
\textbf{Tipo} & Requisito  \\
\hline
\textbf{Fuente del Requisito} & Analista \\
\hline
\textbf{Prioridad} & Alta/Esencial  \\ \hline
\end{tabular}
\caption{Requisito funcional 06}
\label{tab:RF-06}
\end{center}
\end{table}

\begin{table}[H]
\begin{center}
\begin{tabular}{p{3,5cm} p{7cm}}
\multicolumn{2}{c}{\textbf{Requisito: RF-07} } \\
\hline \hline
\textbf{Nombre del Requisito} &  Vibración del volante por diferencia de trayectorias\\
\hline
\textbf{Descripción} & El sistema hará vibrar el volante cuando se detecte una desviación superior o igual al 10\% entre las trayectorias del vehículo y la carretera. \\
\hline
\textbf{Tipo} & Requisito  \\
\hline
\textbf{Fuente del Requisito} & Analista \\
\hline
\textbf{Prioridad} & Alta/Esencial  \\ \hline
\end{tabular}
\caption{Requisito funcional 07}
\label{tab:RF-07}
\end{center}
\end{table}

\begin{table}[H]
\begin{center}
\begin{tabular}{p{3,5cm} p{7cm}}
\multicolumn{2}{c}{\textbf{Requisito: RF-08} } \\
\hline \hline
\textbf{Nombre del Requisito} &  Excepción de aviso de cambio de carril por activación del intermitente\\
\hline
\textbf{Descripción} & El sistema de aviso de cambio de carril se desactivará mientras alguno de los intermitentes se encuentres activos. \\
\hline
\textbf{Tipo} & Requisito  \\
\hline
\textbf{Fuente del Requisito} & Analista  \\
\hline
\textbf{Prioridad} & Alta/Esencial  \\ \hline
\end{tabular}
\caption{Requisito funcional 08}
\label{tab:RF-08}
\end{center}
\end{table}

\begin{table}[H]
\begin{center}
\begin{tabular}{p{3,5cm} p{7cm}}
\multicolumn{2}{c}{\textbf{Requisito: RF-09} } \\
\hline \hline
\textbf{Nombre del Requisito} &  Corrección de la dirección \\
\hline
\textbf{Descripción} & El sistema corregirá un máximo de 5º la dirección del volante cuando detecte que hay una desviación entre las trayectorias superior al vehículo una desviación en la trayectoria superior al 15\%. \\
\hline
\textbf{Tipo} & Requisito  \\
\hline
\textbf{Fuente del Requisito} & Analista  \\
\hline
\textbf{Prioridad} & Alta/Esencial  \\ \hline
\end{tabular}
\caption{Requisito funcional 09}
\label{tab:RF-09}
\end{center}
\end{table}

\newpage

\subsubsection{Alerta de velocidad}

\begin{table}[H]
\begin{center}
\begin{tabular}{p{3,5cm} p{7cm}}
\multicolumn{2}{c}{\textbf{Requisito: RF-10} } \\
\hline \hline
\textbf{Nombre del Requisito} & Reconocimiento de las señales de tráfico \\
\hline
\textbf{Descripción} & El sistema detectará las señales de tráfico a través de las imágenes recogidas por el hardware. \\
\hline
\textbf{Tipo} & Requisito  \\
\hline
\textbf{Fuente del Requisito} & Analista  \\
\hline
\textbf{Prioridad} & Alta/Esencial  \\ \hline
\end{tabular}
\caption{Requisito funcional 10}
\label{tab:RF-10}
\end{center}
\end{table}

\begin{table}[H]
\begin{center}
\begin{tabular}{p{3,5cm} p{7cm}}
\multicolumn{2}{c}{\textbf{Requisito: RF-11} } \\
\hline \hline
\textbf{Nombre del Requisito} &  Clasificación por formas y colores\\
\hline
\textbf{Descripción} & El sistema clasificará las señales obtenidas de las imágenes por formas y colores.\\
\hline
\textbf{Tipo} & Requisito \\
\hline
\textbf{Fuente del Requisito} & Analista \\
\hline
\textbf{Prioridad} & Alta/Esencial \\ \hline
\end{tabular}
\caption{Requisito funcional 11}
\label{tab:RF-11}
\end{center}
\end{table}

\begin{table}[H]
\begin{center}
\begin{tabular}{p{3,5cm} p{7cm}}
\multicolumn{2}{c}{\textbf{Requisito: RF-12} } \\
\hline \hline
\textbf{Nombre del Requisito} &  Reconocimiento señales de velocidad \\
\hline
\textbf{Descripción} & El sistema reconocerá cuales de las señales detectadas en las imágenes se corresponden con señales de velocidad.
\\
\hline
\textbf{Tipo} & Requisito \\
\hline
\textbf{Fuente del Requisito} & Analista \\
\hline
\textbf{Prioridad} & Alta/Esencial \\ \hline
\end{tabular}
\caption{Requisito funcional 12}
\label{tab:RF-12}
\end{center}
\end{table}

\begin{table}[H]
\begin{center}
\begin{tabular}{p{3,5cm} p{7cm}}
\multicolumn{2}{c}{\textbf{Requisito: RF-13} } \\
\hline \hline
\textbf{Nombre del Requisito} & Determinación velocidad máxima a partir de las señales \\
\hline
\textbf{Descripción} & El sistema determinará la velocidad máxima a la que se puede circular a partir de la información obtenida de las señales.\\
\hline
\textbf{Tipo} & Requisito \\
\hline
\textbf{Fuente del Requisito} & Cliente \\
\hline
\textbf{Prioridad} & Alta/Esencial \\ \hline
\end{tabular}
\caption{Requisito funcional 13}
\label{tab:RF-13}
\end{center}
\end{table}

\begin{table}[H]
\begin{center}
\begin{tabular}{p{3,5cm} p{7cm}}
\multicolumn{2}{c}{\textbf{Requisito: RF-14} } \\
\hline \hline
\textbf{Nombre del Requisito} & Determinación velocidad mínima \\
\hline
\textbf{Descripción} &  El sistema determinará la velocidad mínima a la que está permitido circular en esa vía a partir de la velocidad máxima.\\
\hline
\textbf{Tipo} & Requisito \\
\hline
\textbf{Fuente del Requisito} & Analista \\
\hline
\textbf{Prioridad} & Alta/Esencial \\ \hline
\end{tabular}
\caption{Requisito funcional 14}
\label{tab:RF-14}
\end{center}
\end{table}

\begin{table}[H]
\begin{center}
\begin{tabular}{p{3,5cm} p{7cm}}
\multicolumn{2}{c}{\textbf{Requisito: RF-15} } \\
\hline \hline
\textbf{Nombre del Requisito} & Determinación tipo de vía \\
\hline
\textbf{Descripción} & El sistema utilizará la posición GPS y los mapas para determinar el tipo de vía en la que se encuentra el vehículo. \\
\hline
\textbf{Tipo} & Requisito \\
\hline
\textbf{Fuente del Requisito} & Analista \\
\hline
\textbf{Prioridad} & Alta/Esencial \\ \hline
\end{tabular}
\caption{Requisito funcional 15}
\label{tab:RF-15}
\end{center}
\end{table}

\begin{table}[H]
\begin{center}
\begin{tabular}{p{3,5cm} p{7cm}}
\multicolumn{2}{c}{\textbf{Requisito: RF-16} } \\
\hline \hline
\textbf{Nombre de Requisito} & Velocidades máximas de la vía según el código de circulación  \\
\hline
\textbf{Descripción} & El sistema dispondrá de las velocidades máximas permitidas para cada tipo de vía según el código de circulación del país en el que el vehículo se encuentre. \\
\hline
\textbf{Tipo} & Requisito \\
\hline
\textbf{Fuente del Requisito} & Analista \\
\hline
\textbf{Prioridad} & Alta/Esencial \\ \hline
\end{tabular}
\caption{Requisito funcional 16}
\label{tab:RF-16}
\end{center}
\end{table}


\begin{table}[H]
\begin{center}
\begin{tabular}{p{3,5cm} p{7cm}}
\multicolumn{2}{c}{\textbf{Requisito: RF-17} } \\
\hline \hline
\textbf{Nombre de Requisito} & Velocidad máxima al cambiar de vía  \\
\hline
\textbf{Descripción} & Por defecto, cuando se produce un cambio de tipo de vía, el sistema establecerá como velocidad máxima la obtenida por el código de circulación en función del tipo de vía.\\
\hline
\textbf{Tipo} & Requisito \\
\hline
\textbf{Fuente del Requisito} & Analista \\
\hline
\textbf{Prioridad} & Alta/Esencial \\ \hline
\end{tabular}
\caption{Requisito funcional 17}
\label{tab:RF-17}
\end{center}
\end{table}

\begin{table}[H]
\begin{center}
\begin{tabular}{p{3,5cm} p{7cm}}
\multicolumn{2}{c}{\textbf{Requisito: RF-18} } \\
\hline \hline
\textbf{Nombre del Requisito} & Actualización de velocidad máxima \\
\hline
\textbf{Descripción} & Cuando el sistema detecte señales de velocidad verticales o luminosas, se actualizará la velocidad máxima con la información obtenida a través de estas. \\
\hline
\textbf{Tipo} & Requisito \\
\hline
\textbf{Fuente del Requisito} & Analista \\
\hline
\textbf{Prioridad} & Alta/Esencial \\ \hline
\end{tabular}
\caption{Requisito funcional 18}
\label{tab:RF-18}
\end{center}
\end{table}

\begin{table}[H]
\begin{center}
\begin{tabular}{p{3,5cm} p{7cm}}
\multicolumn{2}{c}{\textbf{Requisito: RF-19} } \\
\hline \hline
\textbf{Nombre de Requisito} & Priorización señales luminosas \\
\hline
\textbf{Descripción} & El sistema priorizará la información recibida por las señales luminosas frente a las verticales. \\
\hline
\textbf{Tipo} & Requisito \\
\hline
\textbf{Fuente del Requisito} & Analista \\
\hline
\textbf{Prioridad} & Alta/Esencial \\ \hline
\end{tabular}
\caption{Requisito funcional 19}
\label{tab:RF-19}
\end{center}
\end{table}

\begin{table}[H]
\begin{center}
\begin{tabular}{p{3,5cm} p{7cm}}
\multicolumn{2}{c}{\textbf{Requisito: RF-20} } \\
\hline \hline
\textbf{Nombre de Requisito} & Priorización señales de obra\\
\hline
\textbf{Descripción} & El sistema priorizará la información recibida por las señales de velocidad con un fondo amarillo frente a cualquier otra señal de velocidad. \\
\hline
\textbf{Tipo} & Requisito \\
\hline
\textbf{Fuente del Requisito} & Analista \\
\hline
\textbf{Prioridad} & Alta/Esencial \\ \hline
\end{tabular}
\caption{Requisito funcional 20}
\label{tab:RF-20}
\end{center}
\end{table}

\begin{table}[H]
\begin{center}
\begin{tabular}{p{3,5cm} p{7cm}}
\multicolumn{2}{c}{\textbf{Requisito: RF-21} } \\
\hline \hline
\textbf{Nombre del Requisito} & Restauración de la velocidad por defecto \\
\hline
\textbf{Descripción} & Si el sistema no detecta una señal de velocidad en un tiempo superior a 5 minutos, se restaurará como velocidad máxima la obtenida en función del tipo de vía y la posición GPS.  \\
\hline
\textbf{Tipo} & Requisito  \\
\hline
\textbf{Fuente del Requisito} & Analista  \\
\hline
\textbf{Prioridad} & Media/Deseado  \\ \hline
\end{tabular}
\caption{Requisito funcional 21}
\label{tab:RF-21}
\end{center}
\end{table}

\begin{table}[H]
\begin{center}
\begin{tabular}{p{3,5cm} p{7cm}}
\multicolumn{2}{c}{\textbf{Requisito: RF-22} } \\
\hline \hline
\textbf{Nombre del Requisito} & Configuración límite de velocidad \\
\hline
\textbf{Descripción} & El usuario podrá configurar opcionalmente que el coche no supere en ningún caso el límite de velocidad. \\
\hline
\textbf{Tipo} & Requisito  \\
\hline
\textbf{Fuente del Requisito} & Analista  \\
\hline
\textbf{Prioridad} & Media/Deseado  \\ \hline
\end{tabular}
\caption{Requisito funcional 22}
\label{tab:RF-22}
\end{center}
\end{table}

\begin{table}[H]
\begin{center}
\begin{tabular}{p{3,5cm} p{7cm}}
\multicolumn{2}{c}{\textbf{Requisito: RF-23} } \\
\hline \hline
\textbf{Nombre del Requisito} & Descativación alerta por exceso de velocidad \\
\hline
\textbf{Descripción} & El usuario podrá desactivar la notificación del exceso de velocidad. \\
\hline
\textbf{Tipo} & Requisito  \\
\hline
\textbf{Fuente del Requisito} & Analista  \\
\hline
\textbf{Prioridad} & Media/Deseado  \\ \hline
\end{tabular}
\caption{Requisito funcional 23}
\label{tab:RF-23}
\end{center}
\end{table}

\begin{table}[H]
\begin{center}
\begin{tabular}{p{3,5cm} p{7cm}}
\multicolumn{2}{c}{\textbf{Requisito: RF-24} } \\
\hline \hline
\textbf{Nombre del Requisito} & Reactivación alerta por exceso de  velocidad \\
\hline
\textbf{Descripción} & Si el usuario ha desactivado el sistema, este volverá a activarse al volver a inciar al sistema, con el arranque del vehículo. \\
\hline
\textbf{Tipo} & Requisito  \\
\hline
\textbf{Fuente del Requisito} & Analista  \\
\hline
\textbf{Prioridad} & Media/Deseado  \\ \hline
\end{tabular}
\caption{Requisito funcional 24}
\label{tab:RF-24}
\end{center}
\end{table}

\clearpage
\subsubsection{Sistema de alerta por pérdida de atención}

\begin{table}[H]
\begin{center}
\begin{tabular}{p{3,5cm} p{7cm}}
\multicolumn{2}{c}{\textbf{Requisito: RF-25} } \\
\hline \hline
\textbf{Nombre del Requisito} & Detección conductor dormido\\
\hline
\textbf{Descripción} & El sistema detectará si el conductor se ha quedado dormido.\\
\hline
\textbf{Tipo} & Requisito  \\
\hline
\textbf{Fuente del Requisito} & Cliente  \\
\hline
\textbf{Prioridad} & Alta/Esencial  \\ \hline
\end{tabular}
\caption{Requisito funcional 25}
\label{tab:RF-25}
\end{center}
\end{table}



\begin{table}[H]
\begin{center}
\begin{tabular}{p{3,5cm} p{7cm}}
\multicolumn{2}{c}{\textbf{Requisito: RF-26} } \\
\hline \hline
\textbf{Nombre del Requisito} & Análisis del comportamiento\\
\hline
\textbf{Descripción} & El sistema analizará el comportamiento del conductor así como la desviación media entre las trayectorias para detectar si el conductor muestra síntomas de fatiga al volante.\\
\hline
\textbf{Tipo} & Requisito  \\
\hline
\textbf{Fuente del Requisito} & Analista  \\
\hline
\textbf{Prioridad} & Alta/Esencial  \\ \hline
\end{tabular}
\caption{Requisito funcional 26}
\label{tab:RF-26}
\end{center}
\end{table}

\begin{table}[H]
\begin{center}
\begin{tabular}{p{3,5cm} p{7cm}}
\multicolumn{2}{c}{\textbf{Requisito: RF-27} } \\
\hline \hline
\textbf{Nombre del Requisito} & Recomendación de descanso\\
\hline
\textbf{Descripción} & El sistema recomendará al conductor que detenga el vehículo y descanse por un tiempo cuando muestre síntomas de fatiga al volante. \\
\hline
\textbf{Tipo} & Requisito  \\
\hline
\textbf{Fuente del Requisito} & Analista  \\
\hline
\textbf{Prioridad} & Alta/Esencial  \\ \hline
\end{tabular}
\caption{Requisito funcional 27}
\label{tab:RF-27}
\end{center}
\end{table}

\begin{table}[H]
\begin{center}
\begin{tabular}{p{3,5cm} p{7cm}}
\multicolumn{2}{c}{\textbf{Requisito: RF-28} } \\
\hline \hline
\textbf{Nombre del Requisito} & Detención del vehículo por conductor dormido\\
\hline
\textbf{Descripción} & Si el sistema, después de haber emitido la señal para alertar al conductor por un periodo superior a 3 segundos, no detecta que este vuelva a tener la atención en la carretera, detendrá progresivamente el vehículo y activará las luces de emergencia.\\
\hline
\textbf{Tipo} & Requisito  \\
\hline
\textbf{Fuente del Requisito} & Cliente  \\
\hline
\textbf{Prioridad} & Alta/Esencial  \\ \hline
\end{tabular}
\caption{Requisito funcional 28}
\label{tab:RF-28}
\end{center}
\end{table}



\subsubsection{Llamada de emergencia}

\begin{table}[H]
\begin{center}
\begin{tabular}{p{3,5cm} p{7cm}}
\multicolumn{2}{c}{\textbf{Requisito: RF-29} } \\
\hline \hline
\textbf{Nombre del Requisito} & Detección de colisión\\
\hline
\textbf{Descripción} &El sistema detectará si el vehículo se ha visto involucrado en una colisión.\\
\hline
\textbf{Tipo} & Requisito  \\
\hline
\textbf{Fuente del Requisito} & Analista  \\
\hline
\textbf{Prioridad} & Alta/Esencial  \\ \hline
\end{tabular}
\caption{Requisito funcional 29}
\label{tab:RF-29}
\end{center}
\end{table}

\begin{table}[H]
\begin{center}
\begin{tabular}{p{3,5cm} p{7cm}}
\multicolumn{2}{c}{\textbf{Requisito: RF-30} } \\
\hline \hline
\textbf{Nombre del Requisito} & Mensaje de emergencia\\
\hline
\textbf{Descripción} &En caso de colisión, el sistema enviará un mensaje a un centro de llamadas de emergencia.\\
\hline
\textbf{Tipo} & Requisito  \\
\hline
\textbf{Fuente del Requisito} & Cliente  \\
\hline
\textbf{Prioridad} & Alta/Esencial  \\ \hline
\end{tabular}
\caption{Requisito funcional 30}
\label{tab:RF-30}
\end{center}
\end{table}

\begin{table}[H]
\begin{center}
\begin{tabular}{p{3,5cm} p{7cm}}
\multicolumn{2}{c}{\textbf{Requisito: RF-31} } \\
\hline \hline
\textbf{Nombre del Requisito} & Detección del número de ocupantes\\
\hline
\textbf{Descripción} & El sistema detectará el número de ocupantes del vehículo.\\
\hline
\textbf{Tipo} & Requisito  \\
\hline
\textbf{Fuente del Requisito} & Analista  \\
\hline
\textbf{Prioridad} & Alta/Esencial  \\ \hline
\end{tabular}
\caption{Requisito funcional 31}
\label{tab:RF-31}
\end{center}
\end{table}

\begin{table}[H]
\begin{center}
\begin{tabular}{p{3,5cm} p{7cm}}
\multicolumn{2}{c}{\textbf{Requisito: RF-32} } \\
\hline \hline
\textbf{Nombre del Requisito} & Formato del mensaje\\
\hline
\textbf{Descripción} & El mensaje deberá contener la posición GPS del vehículo, así como la velocidad a la que se ha producido el impacto y el número de ocupantes del vehículo.\\
\hline
\textbf{Tipo} & Requisito  \\
\hline
\textbf{Fuente del Requisito} & Analista  \\
\hline
\textbf{Prioridad} & Alta/Esencial  \\ \hline
\end{tabular}
\caption{Requisito funcional 32}
\label{tab:RF-32}
\end{center}
\end{table}


\begin{table}[H]
\begin{center}
\begin{tabular}{p{3,5cm} p{7cm}}
\multicolumn{2}{c}{\textbf{Requisito: RF-33} } \\
\hline \hline
\textbf{Nombre del Requisito} & Activación de las luces de emergencia\\
\hline
\textbf{Descripción} & En caso de colisión, el sistema activará las luces de emergencia.\\
\hline
\textbf{Tipo} & Requisito  \\
\hline
\textbf{Fuente del Requisito} & Analista  \\
\hline
\textbf{Prioridad} & Alta/Esencial  \\ \hline
\end{tabular}
\caption{Requisito funcional 33}
\label{tab:RF-33}
\end{center}
\end{table}

\clearpage
\subsubsection{Alertas de precolisión}

\begin{table}[H]
\begin{center}
\begin{tabular}{p{3,5cm} p{7cm}}
\multicolumn{2}{c}{\textbf{Requisito: RF-34} } \\
\hline \hline
\textbf{Nombre del Requisito} & Detección de obstáculos\\
\hline
\textbf{Descripción} &  El sistema detectará cualquier obstáculo delante del vehículo.\\
\hline
\textbf{Tipo} & Requisito  \\
\hline
\textbf{Fuente del Requisito} & Cliente  \\
\hline
\textbf{Prioridad} & Alta/Esencial \\ \hline
\end{tabular}
\caption{Requisito funcional 34}
\label{tab:RF-34}
\end{center}
\end{table}

\begin{table}[H]
\begin{center}
\begin{tabular}{p{3,5cm} p{7cm}}
\multicolumn{2}{c}{\textbf{Requisito: RF-35} } \\
\hline \hline
\textbf{Nombre del Requisito} & Análisis del riesgo de una colisión\\
\hline
\textbf{Descripción} &  El sistema analizará el riesgo de una colisión con un obstáculo en función de la velocidad a la que circula el vehículo y la variación de distancia con el obstáculo por milisegundo.\\
\hline
\textbf{Tipo} & Requisito  \\
\hline
\textbf{Fuente del Requisito} & Analista  \\
\hline
\textbf{Prioridad} & Alta/Esencial \\ \hline
\end{tabular}
\caption{Requisito funcional 35}
\label{tab:RF-35}
\end{center}
\end{table}

\begin{table}[H]
\begin{center}
\begin{tabular}{p{3,5cm} p{7cm}}
\multicolumn{2}{c}{\textbf{Requisito: RF-36} } \\
\hline \hline
\textbf{Nombre del Requisito} & Generación de la notificación por colisión\\
\hline
\textbf{Descripción} &  El sistema notificará al conductor cuando se prevea que, a la velocidad actual, una colisión es probable al 50\% o más.\\
\hline
\textbf{Tipo} & Requisito  \\
\hline
\textbf{Fuente del Requisito} & Cliente  \\
\hline
\textbf{Prioridad} & Alta/Esencial \\ \hline
\end{tabular}
\caption{Requisito funcional 36}
\label{tab:RF-36}
\end{center}
\end{table}

\begin{table}[H]
\begin{center}
\begin{tabular}{p{3,5cm} p{7cm}}
\multicolumn{2}{c}{\textbf{Requisito: RF-37} } \\
\hline \hline
\textbf{Nombre del Requisito} & Ayuda en el frenado\\
\hline
\textbf{Descripción} & Cuando se haya generado la notificación por colisión y el conductor pise el freno, el sistema frenará lo suficiente como para evitar la colisión. \\
\hline
\textbf{Tipo} & Requisito  \\
\hline
\textbf{Fuente del Requisito} & Analista  \\
\hline
\textbf{Prioridad} & Alta/Esencial \\ \hline
\end{tabular}
\caption{Requisito funcional 37}
\label{tab:RF-37}
\end{center}
\end{table}

\begin{table}[H]
\begin{center}
\begin{tabular}{p{3,5cm} p{7cm}}
\multicolumn{2}{c}{\textbf{Requisito: RF-38} } \\
\hline \hline
\textbf{Nombre del Requisito} & Colisión inminente\\
\hline
\textbf{Descripción} &  El sistema considerará que se va a producir una colisión inminente cuando la probabilidad de evitar el obstaculo frenando el vehículo es menor que 70\%  \\
\hline
\textbf{Fuente del Requisito} & Analista \\
\hline
\textbf{Prioridad} & Alta/Esencial \\ \hline
\end{tabular}
\caption{Requisito funcional 38}
\label{tab:RF-38}
\end{center}
\end{table}

\begin{table}[H]
\begin{center}
\begin{tabular}{p{3,5cm} p{7cm}}
\multicolumn{2}{c}{\textbf{Requisito: RF-39} } \\
\hline \hline
\textbf{Nombre del Requisito} & Primera medida por colisión inminente\\
\hline
\textbf{Descripción} &  Si se prevé una colisión inminente, el sistema reducirá la velocidad\\
\hline
\textbf{Tipo} & Requisito  \\
\hline
\textbf{Fuente del Requisito} & Cliente  \\
\hline
\textbf{Prioridad} & Alta/Esencial \\ \hline
\end{tabular}
\caption{Requisito funcional 39}
\label{tab:RF-39}
\end{center}
\end{table}

\begin{table}[H]
\begin{center}
\begin{tabular}{p{3,5cm} p{7cm}}
\multicolumn{2}{c}{\textbf{Requisito: RF-40} } \\
\hline \hline
\textbf{Nombre del Requisito} & Segunda medida por colisión inminente\\
\hline
\textbf{Descripción} &  Si se prevé una colisión inminente, el sistema ajustará los cinturones de seguridad.\\
\hline
\textbf{Tipo} & Requisito  \\
\hline
\textbf{Fuente del Requisito} & Cliente  \\
\hline
\textbf{Prioridad} & Alta/Esencial \\ \hline
\end{tabular}
\caption{Requisito funcional 40}
\label{tab:RF-40}
\end{center}
\end{table}

\begin{table}[H]
\begin{center}
\begin{tabular}{p{3,5cm} p{7cm}}
\multicolumn{2}{c}{\textbf{Requisito: RF-41} } \\
\hline \hline
\textbf{Nombre del Requisito} & Tercera medida por colisión inminente\\
\hline
\textbf{Descripción} &  Si se prevé una colisión inminente, el sistema fijará los asientos para optimizar el rendimiento de los airbag.\\
\hline
\textbf{Tipo} & Requisito  \\
\hline
\textbf{Fuente del Requisito} & Cliente  \\
\hline
\textbf{Prioridad} & Alta/Esencial \\ \hline
\end{tabular}
\caption{Requisito funcional 41}
\label{tab:RF-41}
\end{center}
\end{table}

\begin{table}[H]
\begin{center}
\begin{tabular}{p{3,5cm} p{7cm}}
\multicolumn{2}{c}{\textbf{Requisito: RF-42} } \\
\hline \hline
\textbf{Nombre del Requisito} & Cuarta medida por colisión inminente\\
\hline
\textbf{Descripción} &  Si se prevé una colisión inminente, el sistema cerrará las ventanillas.\\
\hline
\textbf{Tipo} & Requisito  \\
\hline
\textbf{Fuente del Requisito} & Cliente  \\
\hline
\textbf{Prioridad} & Alta/Esencial \\ \hline
\end{tabular}
\caption{Requisito funcional 42}
\label{tab:RF-42}
\end{center}
\end{table}

\begin{table}[H]
\begin{center}
\begin{tabular}{p{3,5cm} p{7cm}}
\multicolumn{2}{c}{\textbf{Requisito: RF-43} } \\
\hline \hline
\textbf{Nombre del Requisito} & Reversión de las acciones\\
\hline
\textbf{Descripción} &  Si se consigue evitar el accidente, el sistema invertirá las acciones realizadas en la preparación del vehículo para el impacto.\\
\hline
\textbf{Tipo} & Requisito  \\
\hline
\textbf{Fuente del Requisito} & Cliente  \\
\hline
\textbf{Prioridad} & Media/Deseada \\ \hline
\end{tabular}
\caption{Requisito funcional 43}
\label{tab:RF-43}
\end{center}
\end{table}


\subsection{Requisitos no funcionales}

\subsubsection{Requisitos de rendimiento}

\begin{table}[H]
\begin{center}
\begin{tabular}{p{3,5cm} p{7cm}}
\multicolumn{2}{c}{\textbf{Requisito: RNF-01} } \\
\hline \hline
\textbf{Nombre del Requisito} & Tiempo de realización\\
\hline
\textbf{Descripción} &  El sistema debe realizar el 98\% de las operaciones en menos de 0,005 segundos.\\
\hline
\textbf{Tipo} & Restricción  \\
\hline
\textbf{Fuente del Requisito} & Analista  \\
\hline
\textbf{Prioridad} & Media/Deseada \\ \hline
\end{tabular}
\caption{Requisito no funcional 01}
\label{tab:RNF-01}
\end{center}
\end{table}

\begin{table}[H]
\begin{center}
\begin{tabular}{p{3,5cm} p{7cm}}
\multicolumn{2}{c}{\textbf{Requisito: RNF-02} } \\
\hline \hline
\textbf{Nombre del Requisito} & Soporte\\
\hline
\textbf{Descripción} & El sistema debe ser capaz de dar soporte simultaneo a todo el hardware.\\
\hline
\textbf{Tipo} & Restricción  \\
\hline
\textbf{Fuente del Requisito} & Analista  \\
\hline
\textbf{Prioridad} & Media/Deseada \\ \hline
\end{tabular}
\caption{Requisito no funcional 02}
\label{tab:RNF-02}
\end{center}
\end{table}

\begin{table}[H]
\begin{center}
\begin{tabular}{p{3,5cm} p{7cm}}
\multicolumn{2}{c}{\textbf{Requisito: RNF-03} } \\
\hline \hline
\textbf{Nombre del Requisito} & Funcionamiento con baja luminosidad\\
\hline
\textbf{Descripción} & El sistema tiene que funcionar en condiciones de baja luminosidad.\\
\hline
\textbf{Tipo} & Restricción  \\
\hline
\textbf{Fuente del Requisito} & Analista  \\
\hline
\textbf{Prioridad} & Alta/Esencial \\ \hline
\end{tabular}
\caption{Requisito no funcional 03}
\label{tab:RNF-03}
\end{center}
\end{table}

\begin{table}[H]
\begin{center}
\begin{tabular}{p{3,5cm} p{7cm}}
\multicolumn{2}{c}{\textbf{Requisito: RNF-04} } \\
\hline \hline
\textbf{Nombre del Requisito} & Funcionamiento con condiciones adversas\\
\hline
\textbf{Descripción} & El sistema tiene que funcionar en condiciones climatológicas adversas.\\
\hline
\textbf{Tipo} & Restricción  \\
\hline
\textbf{Fuente del Requisito} & Analista  \\
\hline
\textbf{Prioridad} & Alta/Esencial \\ \hline
\end{tabular}
\caption{Requisito no funcional 04}
\label{tab:RNF-04}
\end{center}
\end{table}


\subsubsection{Requisitos de seguridad}

\begin{table}[H]
\begin{center}
\begin{tabular}{p{3,5cm} p{7cm}}
\multicolumn{2}{c}{\textbf{Requisito: RNF-05} } \\
\hline \hline
\textbf{Nombre del Requisito} & Manipulación del software\\
\hline
\textbf{Descripción} & El software que controla al vehículo no puede ser alterado a excepción de personas autorizadas.\\
\hline
\textbf{Tipo} & Restricción  \\
\hline
\textbf{Fuente del Requisito} & Analista  \\
\hline
\textbf{Prioridad} & Alta/Esencial \\ \hline
\end{tabular}
\caption{Requisito no funcional 05}
\label{tab:RNF-05}
\end{center}
\end{table}

\begin{table}[H]
\begin{center}
\begin{tabular}{p{3,5cm} p{7cm}}
\multicolumn{2}{c}{\textbf{Requisito: RNF-06} } \\
\hline \hline
\textbf{Nombre del Requisito} & Cifrado de la información\\
\hline
\textbf{Descripción} & Toda la información personal del usuario deberá ser cifrada con RSA-256.\\
\hline
\textbf{Tipo} & Restricción  \\
\hline
\textbf{Fuente del Requisito} & Analista  \\
\hline
\textbf{Prioridad} & Alta/Esencial \\ \hline
\end{tabular}
\caption{Requisito no funcional 06}
\label{tab:RNF-06}
\end{center}
\end{table}

\subsubsection{Requisitos de fiabilidad}

\begin{table}[H]
\begin{center}
\begin{tabular}{p{3,5cm} p{7cm}}
\multicolumn{2}{c}{\textbf{Requisito: RNF-07} } \\
\hline \hline
\textbf{Nombre del Requisito} & Fiablilidad de reconocimiento de señales.\\
\hline
\textbf{Descripción} &  El software encargado de identificar las señales reconocerá las señales con al menos 85\% de tasa de acierto.\\
\hline
\textbf{Tipo} & Restricción  \\
\hline
\textbf{Fuente del Requisito} & Analista  \\
\hline
\textbf{Prioridad} & Alta/Esencial \\ \hline
\end{tabular}
\caption{Requisito no funcional 07}
\label{tab:RNF-07}
\end{center}
\end{table}

\begin{table}[H]
\begin{center}
\begin{tabular}{p{3,5cm} p{7cm}}
\multicolumn{2}{c}{\textbf{Requisito: RNF-08} } \\
\hline \hline
\textbf{Nombre del Requisito} & Fiabilidad de detección de punto ciego.\\
\hline
\textbf{Descripción} & El software encargado de detectar los coches situados en el punto ciego reconocerá estos casos con al menos un 97\% de tasa de acierto.\\
\hline
\textbf{Tipo} & Restricción  \\
\hline
\textbf{Fuente del Requisito} & Analista  \\
\hline
\textbf{Prioridad} & Alta/Esencial \\ \hline
\end{tabular}
\caption{Requisito no funcional 08}
\label{tab:RNF-08}
\end{center}
\end{table}

\begin{table}[H]
\begin{center}
\begin{tabular}{p{3,5cm} p{7cm}}
\multicolumn{2}{c}{\textbf{Requisito: RNF-09} } \\
\hline \hline
\textbf{Nombre del Requisito} & Fiabilidad de la llamada de emergencia.\\
\hline
\textbf{Descripción} & El software encargado de realizar la e-call deberá responder con un 100\% de tasa de acierto.\\
\hline
\textbf{Tipo} & Restricción  \\
\hline
\textbf{Fuente del Requisito} & Analista  \\
\hline
\textbf{Prioridad} & Alta/Esencial \\ \hline
\end{tabular}
\caption{Requisito no funcional 09}
\label{tab:RNF-09}
\end{center}
\end{table}

\begin{table}[H]
\begin{center}
\begin{tabular}{p{3,5cm} p{7cm}}
\multicolumn{2}{c}{\textbf{Requisito: RNF-10} } \\
\hline \hline
\textbf{Nombre del Requisito} & Fiabilidad de la detección de cambio de carril.\\
\hline
\textbf{Descripción} &  El software encargado de detectar el cambio involuntario de carril reconocerá este hecho con un 87\% de tasa de acierto.\\
\hline
\textbf{Tipo} & Restricción  \\
\hline
\textbf{Fuente del Requisito} & Analista  \\
\hline
\textbf{Prioridad} & Alta/Esencial \\ \hline
\end{tabular}
\caption{Requisito no funcional 10}
\label{tab:RNF-10}
\end{center}
\end{table}

\begin{table}[H]
\begin{center}
\begin{tabular}{p{3,5cm} p{7cm}}
\multicolumn{2}{c}{\textbf{Requisito: RNF-11} } \\
\hline \hline
\textbf{Nombre del Requisito} & Fiabilidad de la detección de falta de atención.\\
\hline
\textbf{Descripción} &  El software encargado de detectar la falta de atención contará con un 93\% de tasa de acierto.\\
\hline
\textbf{Tipo} & Restricción  \\
\hline
\textbf{Fuente del Requisito} & Analista  \\
\hline
\textbf{Prioridad} & Alta/Esencial \\ \hline
\end{tabular}
\caption{Requisito no funcional 11}
\label{tab:RNF-11}
\end{center}
\end{table}

\begin{table}[H]
\begin{center}
\begin{tabular}{p{3,5cm} p{7cm}}
\multicolumn{2}{c}{\textbf{Requisito: RNF-12} } \\
\hline \hline
\textbf{Nombre del Requisito} & Fiabilidad de la detección de colisión.\\
\hline
\textbf{Descripción} & El software encargado de detectar el pre-crash deberá tener un 98\% de tasa de acierto.\\
\hline
\textbf{Tipo} & Restricción  \\
\hline
\textbf{Fuente del Requisito} & Analista  \\
\hline
\textbf{Prioridad} & Alta/Esencial \\ \hline
\end{tabular}
\caption{Requisito no funcional 12}
\label{tab:RNF-12}
\end{center}
\end{table}


\subsubsection{Requisitos de disponibilidad}

\begin{table}[H]
\begin{center}
\begin{tabular}{p{3,5cm} p{7cm}}
\multicolumn{2}{c}{\textbf{Requisito: RNF-13} } \\
\hline \hline
\textbf{Nombre del Requisito} & Disponibilidad del software.\\
\hline
\textbf{Descripción} & El software deberá estar siempre disponible cuando se arranque el vehículo.\\
\hline
\textbf{Tipo} & Restricción  \\
\hline
\textbf{Fuente del Requisito} & Analista  \\
\hline
\textbf{Prioridad} & Alta/Esencial \\ \hline
\end{tabular}
\caption{Requisito no funcional 13}
\label{tab:RNF-13}
\end{center}
\end{table}


\subsubsection{Requisitos de mantenibilidad}

\begin{table}[H]
\begin{center}
\begin{tabular}{p{3,5cm} p{7cm}}
\multicolumn{2}{c}{\textbf{Requisito: RNF-14} } \\
\hline \hline
\textbf{Nombre del Requisito} & Mantenimiento del software.\\
\hline
\textbf{Descripción} & El sistema recibirá labores de mantenimiento cada 6 meses por profesionales cualificados.\\
\hline
\textbf{Tipo} & Restricción  \\
\hline
\textbf{Fuente del Requisito} & Analista  \\
\hline
\textbf{Prioridad} & Media/Deseada \\ \hline
\end{tabular}
\caption{Requisito no funcional 14}
\label{tab:RNF-14}
\end{center}
\end{table}

\begin{table}[H]
\begin{center}
\begin{tabular}{p{3,5cm} p{7cm}}
\multicolumn{2}{c}{\textbf{Requisito: RNF-15} } \\
\hline \hline
\textbf{Nombre del Requisito} & Registro de fallos del sistema.\\
\hline
\textbf{Descripción} & El software contará con un registro que guardará los fallos del sistema.\\
\hline
\textbf{Tipo} & Restricción  \\
\hline
\textbf{Fuente del Requisito} & Analista  \\
\hline
\textbf{Prioridad} & Media/Deseada \\ \hline
\end{tabular}
\caption{Requisito no funcional 15}
\label{tab:RNF-15}
\end{center}
\end{table}

\subsubsection{Otros requisitos}

\begin{table}[H]
\begin{center}
\begin{tabular}{p{3,5cm} p{7cm}}
\multicolumn{2}{c}{\textbf{Requisito: RNF-16} } \\
\hline \hline
\textbf{Nombre del Requisito} & Base de datos de señales.\\
\hline
\textbf{Descripción} & La base de datos de señales se deberá corresponder con las señales de la normativa del país por donde circula el vehículo.\\
\hline
\textbf{Tipo} & Restricción  \\
\hline
\textbf{Fuente del Requisito} & Analista  \\
\hline
\textbf{Prioridad} & Alta/Esencial \\ \hline
\end{tabular}
\caption{Requisito no funcional 16}
\label{tab:RNF-16}
\end{center}
\end{table}

\begin{table}[H]
\begin{center}
\begin{tabular}{p{3,5cm} p{7cm}}
\multicolumn{2}{c}{\textbf{Requisito: RNF-17} } \\
\hline \hline
\textbf{Nombre del Requisito} & Notificaciones en diferentes idiomas.\\
\hline
\textbf{Descripción} & Las notificaciones de los subsistemas deben estar disponibles en: Español, Inglés, Francés, Alemán, Italiano, Chino y Árabe.\\
\hline
\textbf{Tipo} & Restricción  \\
\hline
\textbf{Fuente del Requisito} & Analista  \\
\hline
\textbf{Prioridad} & Alta/Esencial \\ \hline
\end{tabular}
\caption{Requisito no funcional 17}
\label{tab:RNF-17}
\end{center}
\end{table}

\begin{table}[H]
\begin{center}
\begin{tabular}{p{3,5cm} p{7cm}}
\multicolumn{2}{c}{\textbf{Requisito: RNF-18} } \\
\hline \hline
\textbf{Nombre del Requisito} & Ley de Protección de datos.\\
\hline
\textbf{Descripción} & El sistema debe cumplir la ley de protección de datos vigente en el país en el que opere.\\
\hline
\textbf{Tipo} & Requisito  \\
\hline
\textbf{Fuente del Requisito} & Analista  \\
\hline
\textbf{Prioridad} & Media/Deseada \\ \hline
\end{tabular}
\caption{Requisito no funcional 18}
\label{tab:RNF-18}
\end{center}
\end{table}
