\section{Glosario de términos}

\begin{description}[style=multiline, leftmargin=4cm]

  \item[\textbf{Cifrado RSA-256:}] En criptografía, es un sistema criptográfico de clave pública desarrollado en 1977. Es el primer y más utilizado algoritmo de este tipo y es válido tanto para cifrar como para firmar digitalmente.
  \item[\textbf{Protocolo e-call:}] Es un sistema de llamada de emergencia embarcado en el vehículo capaz de conectarse con el punto de atención de llamadas de emergencia más apropiado, en caso de detección de un accidente o en caso de activación manual por los ocupantes del vehículo, a través de la red de comunicación inalámbrica móvil. Este sistema ha sido desarrollado a iniciativa de la Comisión Europea y su objetivo es proporcionar ayuda rápida a los automovilistas implicados en un accidente de tráfico en cualquier parte de la Unión Europea.
  \item[\textbf{Pre-crash:}] Anglicismo usado a lo largo del documento en sustitución de pre-colisión.
  \item[\textbf{Eye Tracking:}] ``Seguimiento de ojos'' en español, es el proceso de evaluar, bien el punto donde se fija la mirada (donde estamos mirando), o el movimiento del ojo en relación con la cabeza. Este proceso es utilizado en la investigación en los sistemas visuales, en psicología, en lingüística cognitiva y en diseño de productos.
  \item[\textbf{System-on-chip (SOC):}] Describe la tendencia cada vez más frecuente de usar tecnologías de fabricación que integran todos o gran parte de los módulos que componen un computador o cualquier otro sistema informático o electrónico en un único circuito integrado o chip.
  \item[\textbf{Field Programmable Gate Array (FPGA):}] Es un dispositivo programable que contiene bloques de lógica cuya interconexión y funcionalidad puede ser configurada ‘in situ’ mediante un lenguaje de descripción especializado. La lógica programable puede reproducir desde funciones tan sencillas como las llevadas a cabo por una puerta lógica o un sistema combinacional hasta complejos sistemas en un chip.
  \item[\textbf{Ethernet:}] Es un estándar de redes de área local para computadores. Su nombre viene del concepto físico de \textit{ether}. Ethernet define las características de cableado y señalización de nivel físico y los formatos de tramas de datos del nivel de enlace de datos del modelo OSI.
  \par Ethernet se tomó como base para la redacción del estándar internacional IEEE 802.3, siendo usualmente tomados como sinónimos.
  \item[\textbf{Graphics Processor Unit (GPU):}] Es un coprocesador dedicado al procesamiento de gráficos u operaciones de coma flotante, para aligerar la carga de trabajo del procesador central en aplicaciones como los videojuegos o aplicaciones 3D interactivas. De esta forma, mientras gran parte de lo relacionado con los gráficos se procesa en la GPU, la unidad central de procesamiento (CPU) puede dedicarse a otro tipo de cálculos (como la inteligencia artificial o los cálculos mecánicos en el caso de los videojuegos).
  \item[\textbf{Bus CAN:}] Es un protocolo de comunicaciones desarrollado por la firma alemana Robert Bosch GmbH, basado en una topología bus para la transmisión de mensajes en entornos distribuidos. Además ofrece una solución a la gestión de la comunicación entre múltiples CPUs (unidades centrales de proceso).

\end{description}
