
% DOCUMENT

\section{Establecimiento del Alcance del Sistema}

\subsection{Valoración del Estudio de la Situación Actual}
\par La empresa CARSAFETY tiene la necesidad de contratar el desarrollo de un sistema avanzado de seguridad en automóviles, el cual debe incluir: un control del punto ciego, un aviso de cambio de carril, una alerta de velocidad, una alerta por pérdida de atención, una llamada automática de emergencia y una alerta de precolisión. Para poder alcanzar correctamente dichas capacidades, será necesario definir cada uno de los objetivos y restricciones que puedan afectar al desarrollo del sistema.
\par Existen varios objetivos que se desean cumplir con el desarrollo del proyecto:

\begin{itemize}[-]
\item Aumentar la seguridad en la conducción.
\item Diferenciar a CARSAFETY de otras empresas gracias a incorporar este novedoso software.
\item Desarrollar un producto software de calidad que se ajuste lo máximo posible a las necesidades del usuario.
\end{itemize}

\par En las primeras conversaciones con el cliente no se han establecido restricciones económicas. Se le ha presentado un presupuesto orientativo para que tenga una noción sobre el coste que tiene desarrollar el software y éste ha sido aprobado. Además, el cliente es consciente y ha sido informado de que si durante el desarrollo del proyecto van surgiendo costes adicionales que no se habían contemplado en un principio, será necesario aumentar los costes iniciales previstos.
\par Durante el desarrollo del software no se espera encontrar ninguna restricción legal que pueda afectar a la implementación del software ya que siempre trabajamos dentro del ámbito legal. En el caso de encontrar algún problema legal durante el desarrollo del proyecto se consultará con abogados y se procederá según sea conveniente.
\par También se han tenido en cuenta las restricciones operativas y técnicas que puedan aparecer. Tras realizar un estudio exhaustivo de las necesidades del cliente, podemos afirmar que, por el momento, no existen restricciones de este ámbito. Si durante el desarrollo del proyecto se detecta que hay restricciones técnicas y operativas se realizarán reuniones con el cliente para intentar solucionarlas y ajustar el proyecto.


\subsection {Identificación del alcance del sistema}
\par En este apartado, se explicarán las capacidades que va a tener el sistema desarrollado para poder resolver correctamente las necesidades que el cliente, CARSAFETY, ha planteado. Debido a que vamos a desarrollar un producto que sólo va a ser implantado en los vehículos que fabrica CARSAFETY, no se plantea la posibilidad de que el sistema sea escalable a otros. Si otras empresas quisieran implementar también el software, bajo la aprobación de CARSAFETY, serían ellas las que tendrían que adaptar sus vehículos al software.
\par El sistema tiene que proporcionar una solución de seguridad al conductor que cubra los siguientes puntos: un control del punto ciego, un aviso de cambio de carril, una alerta de velocidad, una alerta por pérdida de atención, una llamada automática de emergencia y una alerta de precolisión.
\par A continuación, se identifican las capacidades generales que tiene que contener cada uno de los subsistemas definidos:

\begin{itemize}[-]
\item \textbf{Control del punto ciego}
\par Este problema consiste en la imposibilidad identificar a los vehículos que se van acercando, ya que cuando un vehículo se encuentra detrás y al lado de otro automóvil, no es posible identificarlo correctamente con ninguno de los espejos disponibles. Para poder gestionar correctamente el punto ciego, se instalarán cámaras en la parte trasera del automóvil que permitan vigilar lo que sucede en la parte trasera del vehículo. El sistema software analizará cada una de las imágenes de la cámara y en caso de que no haya ningún vehículo en esta zona, se notificará al conductor a través de los espejos retrovisores.
\item \textbf{Cambio de carril}
\par Cuando el conductor se encuentra circulando, de forma inconsciente, hay veces en que se aproxima demasiado al carril contrario, cruzando la línea que lo delimita llegando a invadirlo completamente. Se instalará una cámara detrás del espejo retrovisor del vehículo, la cual enviará al software imágenes de manera continua, y cuando el vehículo se esté aproximando a la línea de delimitación del carril, el software avisará al conductor mediante una vibración en el volante. Para poder determinar si el conductor está cambiando de carril, el software registrará los movimientos del volante así como la velocidad y trayectoria del vehículo. El sistema sólo funcionará cuando el conductor alcance velocidades propias de una autopista (mínimo 60 km/h) y se desactivará si se utiliza el intermitente.
\item \textbf{Alertas de velocidad}
\par Superar el límite de velocidad es una de las principales causas de accidentes de tráfico, ya sea por distracción del conductor o por el desconocimiento del límite de velocidad asociado a la carretera, entre otros. Para poder reducir los accidentes provocados por superar el límite de velocidad, el software reconocerá las imágenes de las señales de tráfico que recibirá procedentes de una cámara colocada en la parte delantera del vehículo, para poder informar constantemente al conductor de cuál es el límite de velocidad de la carretera por la que se encuentra circulando.
\item \textbf{Pérdida de atención}
\par Para evitar que el conductor se quede dormido al volante, cada vez que se encienda el motor, se activará un algoritmo encargado de medir la posición de los párpados del conductor. Además, el motor del coche, enviará una señal al volante para que mida la presión que el conductor ejerce sobre el volante. La cámara será la encargada de decidir si estos valores son anormales y, en tal caso, enviar una señal acústica procedente del altavoz al conductor para que se despierte. Si la señal acústica se activa, pasados treinta milisegundos se vuelvan a tomar los datos. Este proceso puede repetirse hasta tres veces seguidas, de tal manera que, cuando la cámara detecte que los párpados se están abriendo, la señal acústica parará y se avisará al motor que la señal se ha desactivado. Si pasados los tres ciclos y treinta milisegundos, el conductor sigue dormido, el motor se detiene progresivamente y encendiendo las luces de emergencia. Cuando se haya detenido completamente, se activará de forma automática el freno de mano. En el caso de que la cámara considera que los datos son normales, es decir, que el conductor no se ha quedado dormido al volante, el proceso de toma de datos se repetirá cada dos segundos.
\item \textbf{Llamada automática de emergencia}
\par Cuando el vehículo se encuentre involucrado en alguna colisión, el software enviará un mensaje de emergencia automáticamente al “Punto de respuesta de seguridad pública”, el cual contendrá la ubicación de vehículo en un formato europeo estándar. La ubicación del vehículo será proporcionada por el GPS integrado del automóvil.
\item \textbf{Alertas precolisión}
\par El software reducirá la velocidad, ajustará los cinturones de seguridad, cerrará las ventanillas y colocará los asientos en una posición óptima para que los airbag funcionen correctamente, en caso de que los sensores por radar detecten algún obstáculo fijo delante del vehículo. Una vez se haya sobrepasado exitosamente el obstáculo, las medidas preventivas aplicadas se desactivarán.
\end{itemize}

\subsection {Identificación de los interesados en el sistema: Stakeholders}
\par En este apartado se identificarán a aquellas personas, llamadas stakeholders, que están involucradas en el proyecto, o las cuales pueden verse afectadas por el desarrollo e implementación del software de seguridad solicitado por el cliente. La correcta identificación de los stakeholders permitirá especificar su grado de involucración en el proyecto, así como sus necesidades y expectativas dentro del mismo.
\par Los stakeholders principales que se identifican son los siguientes:

\begin{itemize}[-]
\item \textbf{Empresa desarrolladora del servicio}
\par La empresa que es contratada busca maximizar su beneficio con el producto implementado, así como proporcionar al cliente la mayor satisfacción posible mediante su adaptación a sus necesidades. Si el cliente tiene una buena experiencia con la empresa que ha contratado, la empresa se beneficiará positivamente haciendo crecer su reputación respecto a su competencia dentro del sector. La empresa encargada de desarrollar el servicio es S3.
\item \textbf{Equipo de proyecto}
\par Son las personas que se encargan de desarrollar el proyecto, desde la adjudicación del proyecto, hasta que el producto está listo para ser entregado. El equipo de proyecto busca ampliar sus conocimientos dentro de su área de trabajo para poder crecer profesionalmente. Además, se compensará monetariamente su esfuerzo y grado de implicación en el proyecto. El equipo correspondiente a este proyecto es el siguiente:
\begin{itemize}[-]
\item \textbf{Analista de sistemas:}
es la persona encargada de elaborar los requisitos del sistema, de tal manera que se correspondan con las especificaciones establecidas por el cliente que solicita el sistema. El responsable es Daniel González de la Hera.
\item \textbf{Gestión de configuración:}
es el responsable de asegurar que los cambios realizados en el sistema no afecten a la calidad del mismo durante cualquiera de las etapas de desarrollo. El responsable es Juan Abascal Sánchez.
\item \textbf{Responsable de calidad:}
es el responsable de garantizar que el sistema sea consistente, además de definir los estándares de calidad para que el cliente esté satisfecho con el producto. El responsable es Adriana Lima.
\item \textbf{Responsable de pruebas:}
es el responsable de definir las pruebas necesarias para asegurar que el producto satisface los requisitos establecidos, para tratar de reducir el número de fallos y limitaciones del sistema. El responsable es Carlos Olivares Sánchez-Manjavacas.
\item \textbf{Equipo de desarrolladores:}
es el equipo encargado de desarrollar los requisitos establecidos. El equipo de desarrolladores estará compuesto por Carlos Tormo Sánchez e Irina Saik.
\end{itemize}
\item \textbf{Jefe de proyecto}
\par Es la persona responsable de la gestión del proyecto. Se debe encargar de que el cliente se encuentre satisfecho durante todo momento con el proyecto que se está generando, cumpliendo con los costes y los plazos de entrega definidos. Al igual que el equipo de proyecto, el esfuerzo y trabajo del jefe de proyecto se verán recompensados económicamente. El jefe de proyecto para el desarrollo del sistema de seguridad es Alberto García Hernández.
\item \textbf{Cliente}
\par Es la persona que financia y contrata el proyecto. Una vez el proyecto haya sido desarrollado y entregado al cliente, éste se encargará de explotarlo para obtener beneficios económicos y poder recuperar su inversión. El cliente en este proyecto es CARSAFETY.
\item \textbf{Usuario}
\par Es la persona que realmente utilizará el software desarrollado y el que se beneficiará de su uso. Los usuarios serán aquellos que utilicen un automóvil fabricado por CARSAFETY que tenga incorporado el software de seguridad.
\item \textbf{Viandante}
\par Es la persona que viaja a pie por la calle. Esta persona también se beneficiará del software ya que si aumentamos la seguridad en los vehículos, habrá menos posibilidades de que los viandantes sufran accidentes debido a distracciones de los conductores.
\item \textbf{Otros conductores}
\par Los otros conductores que circulen por la vía también se beneficiarán del software, aunque ellos no lo tengan instalado. En la mayoría de accidentes hay dos partes implicadas, por lo que se podrá evitar muchos de los accidentes que se producirían ya que uno de los vehículos tendrá una mejora en la seguridad.
\end{itemize}
