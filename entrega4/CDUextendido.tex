\chapter{Casos de Uso en formato Extendido}
\section{Formato de los Casos de Uso en formato Expandido}
\par A diferencia del formato de los casos de uso de alto nivel de apartado \ref{chap:cdualtonivel}.\ref{sec:CDUaltoFormat} del Capítulo \textit{Casos de Uso de Alto Nivel}, en este caso el formato de los casos de uso expandidos añadirá información adicional. Entre esta información adicional del caso de uso se incluyen, entre otras, las referencias a los requisitos que cubre el caso de uso, el propósito del caso de uso, las interacciones con el o los usuarios y las posibles alternativas de ejecución.
\par De este modo, en la tabla \ref{tab:formatoCDUE} se puede ver el ejemplo del formato que tendrán los casos de uso en formato expandido.

\begin{table}[h]
\begin{center}
\begin{tabular}{p{3,5cm} p{11cm}}
\multicolumn{2}{c}{\textbf{Identificador} } \\ \hline \hline
Nombre &  \\ \hline
Actores &  \\ \hline
Tipo & \\ \hline
Descripción &  \\ \hline
Referencias &  \\ \hline
Propósito &  \\ \hline
Precondiciones &  \\ \hline
Postcondiciones &  \\ \hline
Interacción &  \\ \hline
Alternativas &  \\ \hline
\end{tabular}
\caption{Formato de las tablas de los casos de uso en formato expandido.}
\label{tab:formatoCDUE}
\end{center}
\end{table}

\par Así, a continuación se explican cada uno de los campos de la tabla \ref{tab:formatoCDUE}. Nótese que los campos ya explicados en la tabla \ref{tab:formatoCDU} del apartado \ref{chap:cdualtonivel}.\ref{sec:CDUaltoFormat} del Capítulo \textit{Casos de Uso de alto nivel} no serán explicados en este punto.

\begin{description}[style=multiline, leftmargin=4cm]
  \item[\textbf{Referencias:}]
  \item[\textbf{Propósito:}]
  \item[\textbf{Interacción:}]
  \item[\textbf{Alternativas:}]
\end{description}





\section{Actores}
\par Como se vio en el apartado \ref{chap:dcufinal}.\ref{sec:actors} del Capítulo \textit{Diagrama Final de Casos de uso y Matriz de Trazabilidad}, los actores que se relacionan con el sistema son:
\begin{description}[style=multiline, leftmargin=4cm]
\item[\textbf{Conductor:}] actor humano que conduce el vehículo con el sistema de CARSAFETY.
\item[\textbf{Objeto:}] cualquier elemento, cosa, persona que se encuentre en las inmediaciones del vehículo.
\item[\textbf{GPS:}] conexión GPS con los satélites.
\item[\textbf{Señal:}] señal de trafico.
\item[\textbf{Reloj:}] tick de reloj del sistema, representando el tiempo.
\item[\textbf{Lineas de carril:}] lineas de delimitación del carril.
\end{description}

\par Cabe destacar que, como veremos en el siguiente apartado, que no todos los actores entrarán en juego en todas las iteraciones. El reloj y las líneas de carril no interactuarán con ningún caso de uso definido en el primer ciclo, sino que lo harán en la segunda iteración.




\section{Iteraciones}
\par Para la elaboración de los casos de uso se ha realizado un proceso iterativo que se ha cumplimentado en tres ciclos. Una primera aproximación a este proceso puede verse en el apartado \ref{chap:priorizacion}.\ref{sec:prioCalc} del Capítulo \textit{Priorización}. Así, el proceso se realiza en tres iteraciones como puede verse en las tablas \ref{tab:iteracion1}, \ref{tab:iteracion2} y \ref{tab:iteracion3}. En la primera iteración, las funcionalidades implementadas son
\hyperref[tab:CDU-05]{\textit{Activar notificación por velocidad máxima (CDU-05)}},
\hyperref[tab:CDU-12]{\textit{Preparar vehículo para el impacto (CDU-12)}}, \hyperref[tab:CDU-11]{\textit{Activar notificación por riesgo de colisión (CDU-11)}}, \hyperref[tab:CDU-13]{\textit{Reducir velocidad (CDU-13)}} y \hyperref[tab:CDU-09]{\textit{Detener el vehículo (CDU-09)}}.

\par En la segunda iteración, las funcionalidades implementadas son
\hyperref[tab:CDU-03]{\textit{Hacer vibrar el volante (CDU-03)}},
\hyperref[tab:CDU-04]{\textit{Corregir el volante (CDU-04)}},
\hyperref[tab:CDU-06]{\textit{No superar el límite de velocidad (CDU-06)}} y
\hyperref[tab:CDU-08]{\textit{Activar notificación de descanso (CDU-08)}}.

\par Finalmente, en la tercera iteración, las funcionalidades implementadas son
\hyperref[tab:CDU-01]{\textit{Activar notificación de punto ciego en el retrovisor (CDU-01)}},
\hyperref[tab:CDU-02]{\textit{Activar notificación sonora por punto ciego (CDU-02)}},
\hyperref[tab:CDU-07]{\textit{Desactivar notificaciones (CDU-07)}} y
\hyperref[tab:CDU-10]{\textit{Llamada de emergencia (CDU-10)}}.



\section{Casos de Uso en formato Expandido}
\par Tras la definición y aclaración de las iteraciones realizadas en la elaboración de los casos de uso, es el momento de especificar estos casos de uso de manera más detallada (en formato expandido) en el orden marcado por las iteraciones.
