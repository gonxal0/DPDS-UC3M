%DOCUMENT
\section{Recursos}
\subsection{Currículos}

%Jefe de Proyecto
\begin{table}[!hb]
\begin{center}
\begin{tabular}{p{3,5cm} p{9cm}}
\multicolumn{2}{c}{\Large{\textbf{Jefe de Proyecto}}} \\
\hline
\textbf{Nombre:} & Alberto García\\
\textbf{Dirección:} & Calle Marques de Urquijo 15 3º C\\
\textbf{Fecha de nacimiento:} & 27 de Mayo de 1986 \\
\textbf{Teléfono:} & 643 45 79 06\\
\textbf{E-mail:} & albergar2@gmail.com\\
\hline \hline
\multicolumn{2}{c}{\textbf{Formación} } \\
\hline
\textbf{Universidad Carlos III Madrid:} & (2004-2010) Doble grado en ingeniería informática y administración de empresas  \\
\textbf{University of Warwick:} & (2010-2011) MBA  \\
\hline \hline
\multicolumn{2}{c}{\textbf{Experiencia Profesional} } \\
\hline
\textbf{Mahou-SanMiguel} & (2011 - 2014) Project Manager\\
\textbf{S$^3$} & (2014 - Actualidad) Project Manager\\
\hline \hline
\multicolumn{2}{c}{\textbf{Capacidades} } \\
\hline
\multicolumn{1}{c}{Liderazgo} & \multicolumn{1}{c}{Trabajo en Equipo}\\
\multicolumn{1}{c}{Negociación} & \multicolumn{1}{c}{Comunicación} \\
\hline \hline
\end{tabular}
\caption{Jefe de Proyecto}
\label{tab:jefeDeProyecto}
\end{center}
\end{table}



%Analista de sistema
\begin{table}[!hb]
\begin{center}
\begin{tabular}{p{3,5cm} p{9cm}}
\multicolumn{2}{c}{\Large{\textbf{Analista de sistemas}}} \\
\hline
\textbf{Nombre:} & Daniel González de la Hera\\
\textbf{Dirección:} & Av. de Concha Espina, 1, 28036 Madrid\\
\textbf{Fecha de nacimiento:} & 14 de Octubre de 1987 \\
\textbf{Teléfono:} & 611 92 48 13\\
\textbf{E-mail:} & dgdelahera@gmail.com\\
\hline \hline
\multicolumn{2}{c}{\textbf{Formación} } \\
\hline
\textbf{Universidad Carlos III Madrid:} &  (2004-2010) Doble grado en ingeniería informática y administración de empresas  \\
\textbf{Universitat Oberta de Catalunya:} &  (2011-2013) Máster en Software Libre   \\
\textbf{Fundación Universitaria Iberoamericana:} &  (2014-2016) Doctorado en Proyectos    \\
\hline \hline
\multicolumn{2}{c}{\textbf{Experiencia Profesional} } \\
\hline
\textbf{KPMG :} &  (2009-2010) Prácticas en el ámbito de desarrollo de proyectos software  \\
\textbf{S$^3$} & (2010 - Actualidad) Responsable en el área de análisis de sistemas\\
\hline \hline
\multicolumn{2}{c}{\textbf{Capacidades} } \\
\hline
\multicolumn{1}{c}{Liderazgo} & \multicolumn{1}{c}{Trabajo en Equipo} \\
\hline
\end{tabular}
\caption{Analista de Sistemas}
\label{tab:AnalistadeSistemas}
\end{center}
\end{table}



%Responsable de pruebas
\begin{table}[!hb]
\begin{center}
\begin{tabular}{p{3,5cm} p{9cm}}
\multicolumn{2}{c}{\Large{\textbf{Responsable de Pruebas}}} \\
\hline
\textbf{Nombre:} & Carlos Olivares Sánchez-Manjavacas\\
\textbf{Dirección:} & C/ Almendros 8 (de los), 28821 Madrid\\
\textbf{Fecha de nacimiento:} & 29 de Marzo de 1991 \\
\textbf{Teléfono:} & 624 54 36 11\\
\textbf{E-mail:} & carlos.olivares.sm@ss3.com\\
\hline \hline
\multicolumn{2}{c}{\textbf{Formación} } \\
\hline
\textbf{Universidad Carlos III Madrid:} &  (2008-2012) Doble grado en ingeniería informática y administración de empresas  \\
\textbf{MIT:} &  (2013-2015) Master in Software Engeniring   \\
\multicolumn{2}{c}{\textbf{Experiencia Profesional} } \\
\hline
\textbf{Kingston University :} &  (2010-2012) Prácticas en el ámbito de desarrollo machine learning  \\
\textbf{S$^3$} & (2012 - Actualidad) Responsable en el área de Pruebas \\
\hline \hline
\multicolumn{2}{c}{\textbf{Capacidades} } \\
\hline
\multicolumn{1}{c}{Liderazgo} & \multicolumn{1}{c}{Trabajo en Equipo} \\
\hline
\end{tabular}
\caption{Analista de Sistemas}
\label{tab:AnalistadeSistemas}
\end{center}
\end{table}



%Gestión de la configuración
\begin{table}[!hb]
\begin{center}
\begin{tabular}{p{3,5cm} p{9cm}}
\multicolumn{2}{c}{\Large{\textbf{Gestón de la configuración}}} \\
\hline
\textbf{Nombre:} & Juan Abascal Sánchez\\
\textbf{Dirección:} &  Calle de Salsipuedes 28081, Madrid\\
\textbf{Fecha de nacimiento:} & 27 de Noviembre de 1989 \\
\textbf{Teléfono:} & 632 92 48 11\\
\textbf{E-mail:} & jabascal@sss3.es\\
\hline \hline
\multicolumn{2}{c}{\textbf{Formación} } \\
\hline
\textbf{IEB School} & (2013) Máster en Gestión Ágil de proyectos\\
\textbf{Universidad Carlos III de Madrid} & (2006-2011) Doble grado en ingeniería informática y administración de empresas\\
\hline \hline
\multicolumn{2}{c}{\textbf{Experiencia Profesional} } \\
\hline
\textbf{S$^3$} & (2015 - Actualidad)  Control de Versiones y Calidad\\
\textbf{GitHub Inc.} & (2011 - 2015) Programador Senior \\
\hline \hline
\multicolumn{2}{c}{\textbf{Capacidades} } \\
\hline
\multicolumn{1}{c}{Perfil Técnico} & \multicolumn{1}{c}{Trabajo en equipo} \\
\multicolumn{1}{c}{Buena Comunicación} & \multicolumn{1}{c}{Responsabilidad} \\
\hline
\end{tabular}
\caption{Responsable de Gestón de la Configuración}
\label{tab:gestConfiguracion}
\end{center}
\end{table}

%Responsable de Calidad
\begin{table}[!hb]
\begin{center}
\begin{tabular}{p{3,5cm} p{9cm}}
\multicolumn{2}{c}{\Large{\textbf{Responsable de Calidad}}} \\
\hline
\textbf{Nombre:} & Adriana Lima\\
\textbf{Dirección:} & Calle Falsa 123, 28003, Madrid\\
\textbf{Fecha de nacimiento:} & 12 de Junio de 1981 \\
\textbf{Teléfono:} & 612 345 678\\
\textbf{E-mail:} & a.lima@sss3.es\\
\hline \hline
\multicolumn{2}{c}{\textbf{Formación} } \\
\hline
\textbf{Universidad Politécnica de Madrid} &  (2004-2006) Máster en Ingeniería del Software\\
\textbf{Universidad Carlos III de Madrid} & (1999-2003) Grado en ingeniería informática \\
\hline \hline
\multicolumn{2}{c}{\textbf{Experiencia Profesional} } \\
\hline
\textbf{S$^3$} & (2010 - Actualidad)  Responsable en el área de gestión de la calidad\\
\textbf{Accenture} & (2005 - 2009) Consultora en el área de tecnología \\
\hline \hline
\multicolumn{2}{c}{\textbf{Capacidades} } \\
\hline
\multicolumn{1}{c}{Motivación} & \multicolumn{1}{c}{Trabajo en equipo} \\
\multicolumn{1}{c}{Autonomía} & \multicolumn{1}{c}{Responsabilidad} \\
\hline
\end{tabular}
\caption{Responsable de Calidad}
\label{tab:respCalidad}
\end{center}
\end{table}


%Desarrollador
\begin{table}[!hb]
\begin{center}
\begin{tabular}{p{3,5cm} p{9cm}}
\multicolumn{2}{c}{\Large{\textbf{Programador I}}} \\
\hline
\textbf{Nombre:} & Carlos Tormo Sánchez\\
\textbf{Dirección:} & Calle Mayor 42, Madrid \\
\textbf{Fecha de nacimiento:} & 21 de Marzo de 1985 \\
\textbf{Teléfono:} & 612 34 56 78\\
\textbf{E-mail:} & ctormo@sss3.es\\
\hline \hline
\multicolumn{2}{c}{\textbf{Formación} } \\
\hline
\textbf{Universidad Carlos III Madrid:} & (2003 - 2009) Doble grado en ingeniería informática y administración de empresas  \\
\textbf{Universidad Carlos III Madrid:} & (2010-2012) Máster en tecnologías de la computación aplicadas al sector financiero  \\
\hline \hline
\multicolumn{2}{c}{\textbf{Experiencia Profesional} } \\
\hline
\textbf{La Caixa} & (2007 - 2008) Prácticas universitarias en el área de desarrollo de la aplicación móvil de La Caixa \\
\textbf{Everis} & (2009 - 2010) Contrato a jornada completa en el área de desarrollo web con la herramienta Liferay \\
\textbf{S$^3$} & (2010 - Actualidad) Fundador de la compañía y especializado en el área de desarrollo \\
\hline \hline
\multicolumn{2}{c}{\textbf{Capacidades} } \\
\hline
\multicolumn{1}{c}{Responsable} & \multicolumn{1}{c}{Trabajo en Equipo}\\
\multicolumn{1}{c}{Negociación} & \multicolumn{1}{c}{Compañerismo} \\
\hline \hline
\end{tabular}
\caption{Programador I}
\label{tab:programadorI}
\end{center}
\end{table}


%Desarrolladora
\begin{table}[!hb]
\begin{center}
\begin{tabular}{p{3,5cm} p{9cm}}
\multicolumn{2}{c}{\Large{\textbf{Programadora II}}} \\
\hline
\textbf{Nombre:} & Irina Shayk\\
\textbf{Dirección:} & Calle Lombardía 13, 28003, Madrid\\
\textbf{Fecha de nacimiento:} & 6 de Enero de 1986 \\
\textbf{Teléfono:} & 612 335 348\\
\textbf{E-mail:} & ishayk@sss3.es\\
\hline \hline
\multicolumn{2}{c}{\textbf{Formación} } \\
\hline
\textbf{Universidad de Deusto} &  (2004-2009) Ingeniería Informática\\
\hline \hline
\multicolumn{2}{c}{\textbf{Experiencia Profesional} } \\
\hline
\textbf{S$^3$} & (2010 - Actualidad)  Programadora Senior\\
\hline \hline
\multicolumn{2}{c}{\textbf{Capacidades} } \\
\hline
\multicolumn{1}{c}{Resolución de Problemas} & \multicolumn{1}{c}{Trabajo en Equipo} \\
\hline
\end{tabular}
\caption{Programadora II}
\label{tab:programadoraII}
\end{center}
\end{table}

\clearpage

%7.2
\subsection{Capacidad técnica y de gestión}
\par{Para poder realizar correctamente el proyecto se necesitarán un conjunto de herramientas software y hardware, las cuales se detallan a continuación. }
\par{Todo el material mencionado en este punto, se ha especificado también en el documento de costes, con el precio de sus licencias y las unidades necesarias.}

%7.2.1
\subsubsection{Software}
\par{Se necesitarán las herramientas ofimáticas que proporciona Office tales como Word, Excel, Power Point, Project (para realizar el diagrama de GANT), etc. Para poder escribir el código del proyecto se utilizará el editor de texto Atom. Finalmente, cuando sea necesario trabajar con imágenes o vídeos, se utilizará la herramienta Photoshop.}
\par{Además, también será necesaria la herramienta Trello, la cual será utilizada para que cada miembro del equipo sepa qué tareas tiene asignadas y cual es el estado  de las mismas. }
\par{Para poder tener un recuento de las horas que emplea cada miembro del equipo en el desarrollo del proyecto, se utilizará la herramienta Toggl. Dicho software permite crear varios proyectos compuestos por varios miembros, los cuales serán los encargados de activar el contador cuando se esté trabajando en el proyecto.}
\par{Para que los miembros del equipo puedan comunicarse, se utilizará la herramienta Slack, además de tener contratado el pack Google Apps for Work, el cual ofrecerá un servicio de correo electrónico para todos los miembros. }
\par{Dado que trabajaremos tanto con equipos de la marca Apple como de Microsoft, se utilizarán ambos sistemas operativos, además de Linux. Gracias a la herramienta Github cada miembro del equipo actualizará constantemente su progreso en el proyecto.}

%7.2.2
\subsubsection{Material de Desarrollo}
\par El material para desarrollar que se utilizará en este proyecto consta de 2 tabletas IPAD, dos ordenadores MAC y tres ordenadores HP. Estos equipos serán utilizados únicamente para trabajar en la empresa, pero los integrantes del proyecto también pueden utilizar sus propios equipos si así lo prefieren. En el caso que sea necesario imprimir algún documento relacionado con el proyecto, también se podrá hacer mediante la impresora que se pondrá a disposición de los empleados.
