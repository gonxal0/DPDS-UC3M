\section{Conclusiones}
\par El análisis de la situación presentado recoge la información de las primeras etapas del proyecto, que son muy importantes a la hora de orientarlo bien, aunque supongan poco riesgo. Para llevar a cabo las conclusiones hemos usado la técnica retrospectiva PMI (Plus-Minus-Interesting).
\subsection{Minus}
\par Hemos observado ligeras desviaciones respecto a los plazos estimados, tanto en la planificación de la duración de las tareas como en la planificación de horas por empleado dedicadas a cada una de las tareas. Por otro lado, hemos detectado que hay tareas más complejas que otras, las cuales habíamos sobrevalorado en un principio y eso nos ha llevado a realizar entregas tardías.
\subsection{Interesting}
\par En los próximos días se convocará una reunión con toda la plantilla en la que el Jefe de Proyecto, Alberto García Hernández, analizará el trabajo individual de cada uno de los integrantes, así como las dificultades obtenidas y las razones por las que no hemos sido capaces de cumplir los plazos estimados. A continuación, se actualizará la planificación de tareas para ajustarla a las nuevas estimaciones.
\subsection{Positive}
\par El análisis de riesgo realizado se pondrá en práctica con efecto inmediato para llevar a cabo acciones preventivas que nos ayuden a evitar o a controlar el impacto que puedan llegar a tener.
\bigbreak
\par En definitiva, este documento nos ha servido de gran ayuda a la hora de analizar el trabajo realizado, identificar los problemas que han surgido e intentar solucionarlos para las entregas sucesivas. De esta forma, el proyecto continuará avanzando según lo planificado, manteniendo los estándares de calidad.
