\section{Formato de los casos de uso de alto nivel}

\par Para la representación de los Casos de Uso de Alto Nivel se ha utilizado la tabla \ref{tab:formatoCDU}:

\begin{table}[h]
\begin{center}
\begin{tabular}{p{3,5cm} p{11cm}}
\multicolumn{2}{c}{\textbf{Identificador} } \\ \hline \hline
Nombre &  \\ \hline
Actores &  \\ \hline
Tipo & \\ \hline
Descripción &  \\ \hline
Precondiciones &  \\ \hline
Postcondiciones &  \\ \hline
\end{tabular}
\caption{Formato de las tablas de los casos de uso}
\label{tab:formatoCDU}
\end{center}
\end{table}

\par Vamos a proceder a explicar cada uno de los campos de la tabla \ref{tab:formatoCDU}:

\begin{itemize}[-]
\item \textbf{Identificador:} a cada caso de uso se le ha asignado un código con el formato: CDU-XX, donde XX representa el número del caso de uso. Este codigo es utilizado para identificar a los casos de uso inequívocamente, y no vincula a los casos de uso de ninguna forma.
\item \textbf{Nombre:} todos los casos de uso tendrán asignado un nombre, que resumirá breve y concisamente el propósito del caso de uso en cuestión.
Actores: en este campo se enumerarán los actores involucrados en el caso de uso en cuestión.
\item \textbf{Tipo:} los casos de uso se pueden clasificar según dos criterios:
\begin{itemize}[-]
\item \textbf{Primario, secundario u opcional:} destacando la importancia del caso de uso.
\item \textbf{Esencial o real:} indicando el grado de compromiso con la implementación.
\end{itemize}
\item \textbf{Descripción:} es este campo se describe detalladamente el proceso que sigue cada uno de los casos de uso.
\item \textbf{Precondiciones:} se describirán las condiciones previas a la realización de los casos de uso
\item \textbf{Postcondiciones:} se describirá el estado del sistema una vez se haya producido el caso de uso.
\end{itemize}
