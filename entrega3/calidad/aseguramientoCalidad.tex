\section{Especificación detallada del Plan de Aseguramiento de Calidad para el Sistema de Información}
\subsection{Contenido del Plan de Aseguramiento de Calidad para el Sistema de Información}
\par En los puntos sucesivos del documento se expondrán las tareas detalladas que se van a realizar en el cumplimiento del Plan de Aseguramiento de Calidad para comprobar que la totalidad del proyecto cumple los criterios de calidad necesarios y que se han estimado como indispensables para la realización del proyecto de forma correcta.

Como hemos dicho ya varias veces, se va a diseñar un sistema de gestión de proyectos, y por tanto se deberá comprobar la calidad de este diseño, además de realizar las revisiones pertinentes a los documentos generados durante el ciclo de vida del proyecto.

Las revisiones se irán realizando a medida que se vayan completando fases del proyecto hasta llegar al diseño final y completo del producto.

Los responsables de realizar las revisiones y aceptar la validez de los productos serán Adriana Lima como Responsable de Calidad y Alberto García Hernández como Jefe de Proyecto. Además todos los miembros del equipo de trabajo deberán realizar las revisiones que les sean asignadas por el Jefe de Proyecto y comunicar a las dos personas al cargo del Plan de Aseguramiento de Calidad en caso de encontrar algún fallo.

En los siguientes puntos del documento se detallan las revisiones específicas que se tendrán que realizar en el cumplimiento del Plan de Aseguramiento de Calidad.

Para cada una de las revisiones se deberá añadir un Informe de Auditoría que recoja la aprobación o el rechazo del producto revisado, indicando en caso de ser necesario las causas por las que se rechaza dicho producto.
