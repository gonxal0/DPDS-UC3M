\section{Revisión de la Verificación de la Arquitectura del Sistema}
\subsection{Revisión de la Consistencia entre Productos del Diseño}

\par Adriana Lima como Responsable de Calidad deberá realizar la revisión de la consistencia entre Productos del Diseño. Para ello se seguirán las pautas establecidas por el documento de Revisión Sistemática de Métricas de Diseño Orientado a Objetos de Juan José Olmedilla, ya que el documento de Métrica 3 no está orientado a objetos. Esto implica cumplir las siguientes pautas:
\begin{itemize}
\item \textbf{Funcionalidad}: se comprueba que la capacidad del software debe proveer las funciones que cumplen con las necesidades implícitas y explícitas cuando el mismo es utilizado bajo ciertas condiciones.
\item \textbf{Fiabilidad}: la capacidad del software de mantener un nivel específico de rendimiento bajo determinadas condiciones de uso.
\item \textbf{Usabilidad}:la capacidad del producto software de ser entendido, aprendido, usado y atractivo al usuario, cuando se usa bajo ciertas condiciones.
\item \textbf{Eficiencia}:la capacidad del software de ofrecer el rendimiento apropiado con respecto a la cantidad de recursos utilizados, bajo condiciones prefijadas.
\item \textbf{Mantenibilidad}:la capacidad del producto de ser modificado. Dichas modificaciones pueden incluir correcciones, mejoras o adaptaciones a cambios en el entorno y en los requisitos y especificaciones funcionales.
\item \textbf{Portabilidad}:la capacidad del software de ser trasladado de un entorno (informático) a otro.
\end{itemize}

\par Como resultado de esta revisión, se debe generar un Informe de Auditoría que recoja la aceptación o no de la Arquitectura del Sistema y las causas del rechazo en caso de que se produzca.


\subsection{Registro de la Aceptación de la Arquitectura del Sistema}
\par Cuando se haya realizado la revisión de la Arquitectura del Sistema, deberá registrarse en este documento la aceptación para que quede constancia que dicho diseño ha sido aprobado por el Responsable de Calidad y que por tanto cumple con los requisitos de Calidad establecidos en el Plan de Aseguramiento de Calidad. Además deberá generarse un Informe de Auditoría que en caso de rechazo deberá contener las causas de la no aceptación de la Arquitectura del Sistema.
