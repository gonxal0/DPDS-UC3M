\section{Progreso actual del proyecto}

\par En este apartado se muestra el estado actual del proyecto, pudiendo observarse las tareas finalizadas, en las que se puede ver el esfuerzo estimado y el esfuerzo real dedicado a ellas; las tareas que actualmente se están llevando a cabo y las tareas que aún no han comenzado.

\subsection{Tareas Finalizadas}
\begin{table}[h]
\begin{center}
\begin{tabular}{ l l l l l l l l }

	Tarea & Inicio Estimado & Inicio Real & Fin estimado & Fin real & Duración estimada & Duración real & Desviación \\ \hline \hline
	OFE & 42767 & 42769 & 42769 & 42771 & 3 días & 2 días & 1 \\ \hline
	DCC & 42767 & 42769 & 42769 & 42772 & 3 días & 3 días  & 0 \\ \hline
	EVS & 42772 & 42772 & 42783 & 42784 & 11 días & 12 días & -1 \\ \hline
	IQS & 42783 & 42784 & 42788 & 42790 & 5 días & 6 día & -1 \\ \hline
\end{tabular}
\caption{Tareas Finalizadas.}
\label{tab:Tareas Finalizadas}
\end{center}
\end{table}

\subsection{Actividades en marcha}
\par En este punto se muestran las actividades que se empezaron antes de la creación de este documento y aún siguen desarrollándose, se detalla además la fecha de inicio y la fecha estimada de finalización, así como la duración y la fecha de fin estimado de cada tarea. También se muestra la desviación actual de cada tarea respecto a las estimaciones iniciales. Como puede observarse, en el momento de realización de este documente han empezado a aparecer retrasos (en el informe quincenal de seguimiento) que deben ser solventados con el fin de evitar mayores problemas durante el desarrollo del proyecto.

\begin{table}[h]
\begin{center}
\begin{tabular}{ l l l l l l}

	Tarea & Inicio Estimado & Inicio Real & Fin estimado & Duración estimada & Desviación \\ \hline \hline
	PGCal & 42786 & 42787 & 42797 & 5 días & 1 día \\ \hline
\end{tabular}
\caption{Actividades en marcha.}
\label{tab:Actividades en marcha}
\end{center}
\end{table}


\subsection{Actividades a comenzar}
\par En este apartado están recogidas todas las actividades que aún no se han comenzado del proyecto, así como las fechas de inicio y finalización previstas.
\par La tarea DCS está compuesta por tres iteraciones, las cuales contienen sus correspondientes fases de diseño y análisis, en las cuales se detallarán los casos de uso, diagramas de clases, diagramas de secuencia, y todo lo relacionado en estas fases.

\begin{table}[H]
\begin{center}
\begin{tabular}{ l l l}

	Tarea & Inicio Estimado & Fin estimado \\ \hline \hline
	GConf & 42793 & 42797 \\ \hline
	DCS & 42809 & 42991 \\ \hline
	IAS & 42991 & 42993 \\ \hline
\end{tabular}
\caption{Actividades a comenzar.}
\label{tab:Actividades a comenzar}
\end{center}
\end{table}

\subsection{Grado de avance del proyecto}
\par Por último se muestra el porcentaje del trabajo realizado hasta la fecha, además del estimado que se tuvo en cuenta en el documento de oferta.

\begin{table}[h]
\begin{center}
\begin{tabular}{ p{2cm} p{2cm} p{2cm} p{2cm} p{2cm} p{2cm}}

	Tarea & Fecha de inicio & Fecha de finalización estimada & Fecha de finalización real & Porcentaje estimado & Porcentaje real \\ \hline \hline
	OFE & 42769 & 42769 & 42772 & 1 & 1 \\ \hline
	DCC & 42769 & 42769 & 42772 & 1 & 1 \\ \hline
	EVS & 42772 & 42783 & 42784 & 1 & 1 \\ \hline
	IQS & 42783 & 42788 & 42790 & 1 & 1 \\ \hline
	PGCal & 42787 & 42797 & - & 0.4 & 0.33 \\ \hline
	GConf & 42793 & 42797 & - & 0 & 0 \\ \hline
	DCS & 42809 & 42991 & - & 0 & 0 \\ \hline
	IAS & 42991 & 42993 & - & 0 & 0 \\ \hline
\end{tabular}
\caption{Grado de avance.}
\label{tab:Grado de avance}
\end{center}
\end{table}
