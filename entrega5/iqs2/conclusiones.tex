\section{Conclusiones}
\par El análisis de la situación presentado recoge la información de la segunda etapa del proyecto, en la que se lleva a cabo una estimación y planificación en profundidad, así como el análisis del sistema y su diseño de implementación. Para llevar a cabo las conclusiones hemos usado la técnica retrospectiva PMI (Plus-Minus-Interesting).

\subsection{Plus}
\begin{itemize}
  \item Hemos observado que la planificación de las tareas realizada en un primer momento ha sido bastante aproximada a la duración real.
  \item La duración real de algunas tareas ha sido inferior a lo estimado, lo que ha servido de contrabalanza para recuperar el retraso que habíamos obtenido.
\end{itemize}

\subsection{Minus}
\begin{itemize}
  \item Es fácil observar que las fechas de inicio y finalización reales coinciden en muy pocas ocasiones con las estimadas. Sin embargo, pese a llevar unos días de retraso debidos a ciertos contratiempos, la duración de las tareas estimada concuerda con la real.
  \item En los casos en los que una tarea ha sido realizada de forma más rápida que lo estimado, hemos notado que las sucesivas tareas son realizadas en el tiempo sobrante de la tarea anterior más el tiempo estimado a la tarea en cuestión. Esto puede ser debido a la tranquilidad de los empleados al contar con más días de los planificados.
\end{itemize}

\subsection{Interesting}
\begin{itemize}
  \item En los próximos días se convocará una reunión con toda la plantilla en la que el Jefe de Proyecto, Alberto García Hernández, analizará el trabajo individual de cada uno de los integrantes, así como las dificultades obtenidas y las razones por las que no hemos sido capaces de cumplir los plazos estimados. A continuación, se actualizará la planificación de tareas para ajustarla a las nuevas estimaciones.
  \item Las soluciones que se obtengan como conclusión en esta reunión serán puestas en marcha con carácter inmediato.
\end{itemize}
