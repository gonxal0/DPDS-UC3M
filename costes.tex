\documentclass[10pt,a4paper,oldfontcommands]{plantillaDPDS}

\makepage{./img/s3-cabecera.png}
\pagestyle{myruled}


% DOCUMENT
\begin{document}
\pagecolor{fondo}
\color{principal}

\chapter{Costes}
\clearpage

\section{Introducción}
\par A lo largo de este apartado se procederá a evaluar la estimación de costes que supondrá el desarrollo de este proyecto. Para ello, en primer lugar, se indicará el personal a cargo del proyecto, así como el coste por hora de trabajo de cada uno de ellos. Tras realizar una estimación de horas realizadas por cada uno de los empleados, se estimará el coste total del salario del personal que trabajará en el proyecto.
A\par sí mismo, se hará una estimación del coste del material informático utilizado (tanto el referente al software como al hardware), del material fungible, del material del pruebas, de los viajes y dietas y de los costes indirectos.
\par Con todo ello, se proporcionará una estimación del coste total del proyecto, que será utilizada para el presupuesto del mismo y para el documento de oferta remitida.

\section{Cálculo de Costes}

\subsection{Resumen del personal a cargo}
\par Para el desarrollo del proyecto necesitaremos un total de siete miembros trabajando en el equipo. En este personal deberá haber un Jefe de Proyecto, encargado de liderar el equipo y ofrecer las directrices necesarias para el desarrollo del mismo. Así mismo, se contará con un Analista de Sistemas, un especialista en la Gestión de Configuración, un responsable de pruebas y otro de calidad y dos desarrolladores.
\par Así, en la tabla \ref{tab:personal} se puede observar qué empleados formaran parte del equipo de trabajo de este proyecto y cuál será su salario por hora de trabajo.
\par Por otro lado, en la tabla \ref{tab:repHoras} se puede observar el número de horas realizado por cada miembro del equipo en cada una de las tareas. Para esta estimación \textcolor{red}{RELLENAR}.


\begin{table}[H]
\begin{center}
\begin{tabular}{l l c}

\textbf{CARGO} & \textbf{NOMBRES} & \textbf{COSTE/HORA}\\ \hline \hline
Jefe de Proyecto & Alberto García & 35  \\
Analistas de Sistemas & Daniel González & 30\\
Gestión de Configuración & Juan Abascal & 25\\
Responsable de Calidad & Adriana Lima & 25\\
Responsable de Pruebas & Carlos Olivares & 25\\
Desarrolladores & Carlos Tormo y  Irina Saik & 20\\ \hline \hline
\end{tabular}
\caption{Resumen de personal.}
\label{tab:personal}
\end{center}
\end{table}



\begin{center}
\begin{longtable}{lcccccc}

%HEAD
& \textbf{Análisis}	&	\textbf{Diseño}	&	\textbf{Programación}	&	\textbf{Pruebas}	&	\textbf{Sobrecarga}	&	\textbf{TOTAL} \\
& \textbf{(horas)}	&	\textbf{(horas)}	&	\textbf{(horas)}	&	\textbf{(horas)}	&	\textbf{(horas)}	&	\textbf{(horas)} \\
\hline
\hline
\endfirsthead
& \textbf{Análisis}	&	\textbf{Diseño}	&	\textbf{Programación}	&	\textbf{Pruebas}	&	\textbf{Sobrecarga}	&	\textbf{TOTAL} \\
& \textbf{(horas)}	&	\textbf{(horas)}	&	\textbf{(horas)}	&	\textbf{(horas)}	&	\textbf{(horas)}	&	\textbf{(horas)} \\
\hline
\hline
\endhead

%FOOT
\hline \multicolumn{7}{r}{\textit{Continúa en la siguiente página}} \\
\endfoot
\endlastfoot

%table
Jefe de Proyecto	&	52,05	&	197,80	&	104,10	&	78,08	&	390,39	&	822,42	\\
Analista	&	364,36	&	416,42	&	0,00	&	0,00	&	35,14	&	815,92	\\
Gestión de la Configuración	&	41,64	&	197,80	&	0,00	&	0,00	&	269,37	&	508,81	\\
Responsable de Calidad	&	41,64	&	197,80	&	104,10	&	156,16	&	39,04	&	538,74	\\
Responsable de Pruebas	&	10,41	&	10,41	&	416,42	&	546,55	&	39,04	&	1.022,82	\\
Desarrollador 1 &	5,21	&	10,41	&	728,73	&	0,00	&	3,90	&	748,25	\\
Desarrollador 2	&	5,21	&	10,41	&	728,73	&	0,00	&	3,90	&	748,25	\\	\hline \hline
\textbf{TOTAL}	&	\textbf{520,52}	&	\textbf{1.041,04}	&	\textbf{2.082,08}	&	\textbf{780,78}	&\textbf{780,78}	&	\textbf{5.205,20}	\\	\hline

\caption{Reparto de horas}\\
\label{tab:repHoras}
\end{longtable}
\end{center}




\subsection{Salarios de los empleados}
\par Tras estudiar en el apartado anterior el número de horas que se estima realizará cada uno de los empleados en el proyecto, y sabiendo también el coste de los mismos por hora, a continuación se expone el salario total que percibirá cada uno de los empleados. Es esta la información que contiene la tabla \ref{tab:costePersonal}.

\begin{table}[H]
\begin{center}
\begin{tabular}{l l c c}
\textbf{CARGO} & \textbf{NOMBRE} & \textbf{TOTAL HORAS} & \textbf{COSTE (\euro)}\\ \hline \hline
Jefe de Proyecto & Alberto García & 822,42 & 28.784,76\\
Analista de Sistemas & Daniel González & 815,92 & 24.477,45\\
Gestión de Configuración & Juan Abascal & 508,81 & 12.720,21\\
Responsable de Calidad & Adriana Lima & 538,74 & 13.468,46\\
Responsable de Pruebas & Carlos Olivares & 1.022,82 & 25.570,55\\
Desarrollador 1 & Carlos Tormos & 748,25 & 14.964,95\\
Desarrollador 2 & Irina Saik & 748,25 & 14.964,95\\ \hline \hline
\textbf{TOTAL} & & & \textbf{134.951,32} \\ \hline
\end{tabular}
\caption{Coste de empleados.}
\label{tab:costePersonal}
\end{center}
\end{table}



\subsection{Equipos informáticos}
\par Para el desarrollo del proyecto haremos uso de los equipos informáticos indicados en la tabla \ref{tab:hardware}. En ella se puede ver el coste total de los dispositivos. Sin embargo, al usarlos únicamente durante los seis meses que dura el proyecto, en la misma se indica el coste que supondrá el uso de los mismos durante ese periodo de tiempo (amortización). Para el cálculo de la misma, se ha supuesto que todos los equipos se amortizan en 3 años.

\begin{table}[H]
\begin{center}
\begin{tabular}{l c c c c }
\textbf{DESCRIPCIÓN} & \textbf{UNIDADES} & \textbf{PRECIO UNITARIO (\euro)} & \textbf{TOTAL (\euro)} & \textbf{AMORTIZACIÓN (\euro)}\\ \hline \hline
Tablets IPAD & 2 & 400 & 800 & 133.33\\
Ordenadores MAC & 2 & 1400 & 2.800 & 466,66\\
Ordenadores HP & 3 & 800 & 2.400 & 400\\
Servidor dedicado & 1 & 300\euro/mes & 1.800 & 300\\
Impresora & 1 & 400 & 800 & 133,33\\ \hline \hline
\textbf{TOTAL} & & & \textbf{8.600} & \textbf{1.433,33}\\\hline
\end{tabular}
\caption{Hardware informático.}
\label{tab:hardware}
\end{center}
\end{table}



\subsection{Herramientas del software}
\par Serán necesarias las licencias de los programas indicados en la tabla \ref{tab:software} para el desarrollo del proyecto. Para la parte de desarrollo, se utilizará el control de versiones git mediante el programa GitHub, y el editor de código Atom. Para el desarrollo de la documentación necesaria se utilizará Office. Para la gestión del proyecto y el control de tareas y tiempos se utilizarán los programas Toggle y Trello.
\par Así mismo, para la gestión general del proyecto y la comunicación entre miembros del equipo se usaran las \textit{suits} Google Apps for Work y Slack.


\begin{table}[H]
\begin{center}
\begin{tabular}{l c c c}
\textbf{DESCRIPCIÓN} & \textbf{UNIDADES} & \textbf{PRECIO UNITARIO} & \textbf{TOTAL (\euro)}\\ \hline \hline
Licencias GitHub & 7 & 9\euro/mes & 54\\
Licencias Office365 & 7 & 8,80\euro/mes & 52,8\\
Licencia Toggle & 7 & 9\euro/mes & 54\\
Licencia Trello & 7 & 10\euro/mes & 60\\
Licencia Slack & 7 & 7,5\euro/mes & 45\\
Google Apps for Work & 7 & 4\euro/mes & 24\\
Licencia Photoshop & 3 & 19,99\euro/mes & 119,94\\
Licencia Atom & 7 & 0\euro/mes & 0\\
Licencia Toggle & 7 & 9\euro/mes & 54\\ \hline \hline
\textbf{TOTAL} & & & \textbf{463,74}\\ \hline
\end{tabular}
\caption{Software informático.}
\label{tab:software}
\end{center}
\end{table}



\subsection{Material fungible}
\begin{table}[H]
\begin{center}
\begin{tabular}{l c}
\textbf{DESCRIPCIÓN} & \textbf{TOTAL (\euro)}\\ \hline \hline
Recambios de impresora & 200\\
Material de oficina & 200\\ \hline \hline
\textbf{TOTAL} & \textbf{400}\\ \hline
\end{tabular}
\caption{Material fungible.}
\label{tab:fungible}
\end{center}
\end{table}


\subsection{Material de pruebas}
\begin{table}[H]
\begin{center}
\begin{tabular}{l c}
\textbf{DESCRIPCIÓN} & \textbf{TOTAL (\euro)}\\ \hline \hline
3 cámaras NetGear & 400,71\\
Antena GPS Garmin GA 25MCX & 19,53\\
Sensor de distacia & 19,53\\
Quit Rapsberry Pi 3 & 85\\
Arduino Mega 2560 & 35\\ \hline \hline
\textbf{TOTAL} & \textbf{559,77}\\ \hline
\end{tabular}
\caption{Material de pruebas.}
\label{tab:pruebas}
\end{center}
\end{table}





\subsection{Viajes y dietas}
\begin{table}[H]
\begin{center}
\begin{tabular}{l c}
\textbf{DESCRIPCIÓN} & \textbf{TOTAL (\euro)}\\ \hline \hline
Gasolina & 400\\
Comidas & 3.000\\ \hline \hline
\textbf{TOTAL} & \textbf{3.400}\\ \hline
\end{tabular}
\caption{Viajes y dietas.}
\label{tab:viajes}
\end{center}
\end{table}


\subsection{Costes indirectos}
En la siguiente tabla mostramos los costes indirectos derivados de la oficina, tales como electricidad, agua, alquiler e internet.
\begin{table}[H]
\begin{center}
\begin{tabular}{l c}
\textbf{DESCRIPCIÓN} & \textbf{TOTAL}\\ \hline \hline
Alquiler espacio co-working (Sala Tokio - Impact Hub Madrid)\footnote{Al no disponer de oficinas, se utilizará este espacio como sála de trabajo durante 5 horas diarias a un precio de 40\euro/hora} & 1000\euro/semana\\ \hline \hline
\textbf{TOTAL} & \textbf{21.000\euro}\\ \hline
\end{tabular}
\caption{Costes indirectos.}
\label{tab:indirectos}
\end{center}
\end{table}

\end{document}
