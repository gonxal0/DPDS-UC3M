\documentclass[10pt,a4paper,oldfontcommands]{dpds}

\makepage{logouc3m}
\pagestyle{myruled}


% DOCUMENT
\begin{document}
\pagecolor{fondo}
\color{principal}

\chapter{Costes}
\clearpage

\section{Introducción}


\section{Cálculo de Costes}
\subsection{Resumen del personal a cargo}

\subsubsection{Probando}


\begin{table}[H]
\begin{center}
\begin{tabular}{l l c}
\hline
\textbf{CARGO} & \textbf{NOMBRES} & \textbf{COSTE/HORA}\\ \hline
Jefe de Proyecto & Alberto García & 45  \\
Analistas de Sistemas & Daniel González & 35\\
Gestión de Configuración & Juan Abascal & 30\\
Responsable de Calidad & Adriana Lima & 30\\
Responsable de Pruebas & Carlos Olivares & 30\\
Desarrolladores & Carlos Tormo y  Irina Saik & 25\\
\end{tabular}
\caption{Resumen de personal.}
\label{tab:personal}
\end{center}
\end{table}



\begin{center}
\begin{longtable}{|p{1.5cm}|p{1.5cm}|p{1.5cm}|p{1.5cm}|p{1.5cm}|p{1.5cm}|p{1.5cm}|p{1.5cm}|p{1.5cm}|}

%HEAD
\hline
ACTIVIDADES & JEFE DE PROYECTO & ANALISTA 1 & ANALISTA 2 & ANALISTA 3 & CONFIGURACIÓN & CALIDAD & PRUEBAS & TOTAL \\
\hline
\endfirsthead
\multicolumn{4}{c}%
{\tablename\ \thetable\ -- \textit{Continued from previous page}} \\
\hline
ACTIVIDADES & JEFE DE PROYECTO & ANALISTA 1 & ANALISTA 2 & ANALISTA 3 & CONFIGURACIÓN & CALIDAD & PRUEBAS & TOTAL \\
\hline
\endhead

%FOOT
\hline \multicolumn{9}{r}{\textit{Continued on next page}} \\
\endfoot
\hline
\endlastfoot

%table


\caption{Reparto de horas}\\
\end{longtable}
\end{center}




\subsection{Salarios de los empleados}
\begin{table}[H]
\begin{center}
\begin{tabular}{l l c c}
\hline
\textbf{CARGO} & \textbf{NOMBRE} & \textbf{TOTAL HORAS} & \textbf{COSTE}\\ \hline \hline
Jefe de Proyecto & Alberto García & 100 & 4.500\\
Analista de Sistemas & Daniel González & 124 & 4.340\\
Gestión de Configuración & Juan Abascal & 89 & 2.670\\
Responsable de Calidad & Adriana Lima & 93 & 2.790\\
Responsable de Pruebas & Carlos Olivares & 97 & 2.910\\
Desarrollador 1 & Carlos Tormos & 75 & 1.875\\
Desarrollador 2 & Irina Saik & 75 & 1.875\\ \hline \hline
\textbf{TOTAL} & & \textbf{653} & \textbf{20.960} \\ \hline
\end{tabular}
\caption{Coste de empleados.}
\label{tab:costePersonal}
\end{center}
\end{table}



\subsection{Equipos informáticos}
\begin{table}[H]
\begin{center}
\begin{tabular}{l c c c}
\hline
\textbf{DESCRIPCIÓN} & \textbf{UNIDADES} & \textbf{PRECIO UNITARIO} & \textbf{TOTAL}\\ \hline \hline
Tablets IPAD & 2 & 400 & 800\\
Ordenadores MAC & 2 & 1400 & 2800\\
Ordenadores HP & 3 & 800 & 2400\\
Servido dedicado & 1 & 300\euro/mes & \\
Impresora & 1 & 400 & 800\\ \hline \hline
\textbf{TOTAL} & & & \\ \hline
\end{tabular}
\caption{Hardware informático.}
\label{tab:hardware}
\end{center}
\end{table}



\subsection{Herramientas del software}
\begin{table}[H]
\begin{center}
\begin{tabular}{l c c c}
\hline
\textbf{DESCRIPCIÓN} & \textbf{UNIDADES} & \textbf{PRECIO UNITARIO} & \textbf{TOTAL}\\ \hline \hline
Licencias GitHub & 7 & 9\euro/mes & \\
Licencias Office365 & 7 & 8,80\euro/mes & \\
Licencia Toggle & 7 & 9\euro/mes & \\
Licencia Trello & 7 & 10\euro/mes & \\
Licencia Slack & 7 & 7,5\euro/mes & \\
Google Apps for Work & 7 & 4\euro/mes & \\
Licencia Photoshop & 3 & 19,99\euro/mes & \\
Licencia Atom & 7 & 0\euro/mes & 0\\
Licencia Toggle & 7 & 9\euro/mes & \\ \hline \hline
\textbf{TOTAL} & & & \\ \hline
\end{tabular}
\caption{Software informático.}
\label{tab:software}
\end{center}
\end{table}



\subsection{Material fungible}
\begin{table}[H]
\begin{center}
\begin{tabular}{l c}
\hline
\textbf{DESCRIPCIÓN} & \textbf{TOTAL}\\ \hline \hline
Recambios de impresora & 200\\
Material de oficina & 200\\ \hline \hline
\textbf{TOTAL} & \\ \hline
\end{tabular}
\caption{Material fungible.}
\label{tab:fungible}
\end{center}
\end{table}


\subsection{Material de pruebas}
\begin{table}[H]
\begin{center}
\begin{tabular}{l c}
\hline
\textbf{DESCRIPCIÓN} & \textbf{TOTAL}\\ \hline \hline
Seat Ibiza 2007 & 3.000\\
3 cámaras NetGear & 400,71\\
Antena GPS Garmin GA 25MCX & 19,53\\
Sensor de distacia & 19,53\\
Quit Rapsberry Pi 3 & 85\\
Arduino Mega 2560 & 35\\ \hline \hline
\textbf{TOTAL} & \\ \hline
\end{tabular}
\caption{Material de pruebas.}
\label{tab:pruebas}
\end{center}
\end{table}





\subsection{Viajes y dietas}
\begin{table}[H]
\begin{center}
\begin{tabular}{l c}
\hline
\textbf{DESCRIPCIÓN} & \textbf{TOTAL}\\ \hline \hline
Gasolina & 400\\
Comidas & 3.000\\ \hline \hline
\textbf{TOTAL} & \\ \hline
\end{tabular}
\caption{Viajes y dietas.}
\label{tab:viajes}
\end{center}
\end{table}


\subsection{Costes indirectos}
En la siguiente tabla mostramos los costes indirectos derivados de la oficina, tales como electricidad, agua, alquiler e internet.
\begin{table}[H]
\begin{center}
\begin{tabular}{l c}
\hline
\textbf{DESCRIPCIÓN} & \textbf{TOTAL}\\ \hline \hline
Alquiler espacio co-working (Sala Tokio - Impact Hub Madrid)\footnote{Al no disponer de oficinas, se utilizará este espacio como sála de trabajo durante 5 horas diarias a un precio de 40\euro/hora} & 1000\euro/semana\\ \hline \hline
\textbf{TOTAL} & \\ \hline
\end{tabular}
\caption{Costes indirectos.}
\label{tab:indirectos}
\end{center}
\end{table}


\subsection{Resumenes totales}
En esta tabla aparece el sumatorio de todos los sub-totales anteriormente calculados.
En la siguiente tabla mostramos los costes indirectos derivados de la oficina, tales como electricidad, agua, alquiler e internet.
\begin{table}[H]
\begin{center}
\begin{tabular}{l c}
\hline
\textbf{DESCRIPCIÓN} & \textbf{TOTAL}\\ \hline \hline
Sueldo del equipo de trabajo & 200\\
Equipos informáticos & 200\\
Software informático & 200\\
Material fungible & 200\\
Viajes y dietas & 200\\
Costes informáticos & 200\\ \hline \hline
\textbf{TOTAL} & \\ \hline
\end{tabular}
\caption{Resúmen de costes totales.}
\label{tab:resumenTotal}
\end{center}
\end{table}



\subsection{Totales sin IVA}
En esta tabla se muestra el coste del proyecto sin I.V.A, así como, el riesgo y el beneficio a obtener por la empresa.
\begin{table}[H]
\begin{center}
\begin{tabular}{l c}
\hline
\textbf{DESCRIPCIÓN} & \textbf{TOTAL}\\ \hline \hline
Coste del proyecto (sin IVA) & 200\\
Riesgo (en porcentaje) & 15\% \\
Beneficio (en porcentaje)** & 10\% \\ \hline \hline
\textbf{TOTAL} & \\ \hline
\end{tabular}
\caption{Riesgos y beneficios.}
\label{tab:total}
\end{center}
\end{table}



\end{document}
