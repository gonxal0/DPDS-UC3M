
\chapter{Costes}\label{chap:costes}
\clearpage

\section{Introducción}
\par A lo largo de este apartado se procederá a evaluar la estimación de costes que supondrá el desarrollo de este proyecto. Para ello, en primer lugar, se indicará el personal a cargo del proyecto, así como el coste por hora de trabajo de cada uno de ellos. Tras realizar una estimación de horas realizadas por cada uno de los empleados, se estimará el coste total del salario del personal que trabajará en el proyecto.
\par Así mismo, se hará una estimación del coste del material informático utilizado (tanto el referente al software como al hardware), del material fungible, del material del pruebas, de los viajes y dietas y de los costes indirectos.
\par Con todo ello, se proporcionará una estimación del coste total del proyecto, que será utilizada para el presupuesto del mismo y para el documento de oferta remitida.

\section{Cálculo de Costes}

\subsection{Resumen del personal a cargo}
\par Para el desarrollo del proyecto necesitaremos un total de siete miembros trabajando en el equipo. En este personal deberá haber un Jefe de Proyecto, encargado de liderar el equipo y ofrecer las directrices necesarias para el desarrollo del mismo. Así mismo, se contará con un Analista de Sistemas, un especialista en la Gestión de Configuración, un responsable de pruebas y otro de calidad y dos desarrolladores.
\par Así, en la tabla \ref{tab:personal} se puede observar qué empleados formaran parte del equipo de trabajo de este proyecto y cuál será su salario por hora de trabajo.
\par Por otro lado, en la tabla \ref{tab:repHoras} se puede observar el número de horas realizado por cada miembro del equipo en cada una de las tareas.
\par Para esta estimación se ha utilizado la estimación mediante casos de uso. Puede verse más información en \ref{sec:apEst}.

\begin{table}[H]
\begin{center}
\begin{tabular}{l l c}

\textbf{CARGO} & \textbf{NOMBRES} & \textbf{COSTE/HORA}\\ \hline \hline
Jefe de Proyecto & Alberto García & 35  \\
Analistas de Sistemas & Daniel González & 30\\
Gestión de Configuración & Juan Abascal & 25\\
Responsable de Calidad & Adriana Lima & 25\\
Responsable de Pruebas & Carlos Olivares & 25\\
Desarrolladores & Carlos Tormo y  Irina Saik & 20\\ \hline \hline
\end{tabular}
\caption{Resumen de personal.}
\label{tab:personal}
\end{center}
\end{table}



\begin{center}
\begin{longtable}{lcccccc}

%HEAD
& \textbf{Análisis}	&	\textbf{Diseño}	&	\textbf{Programación}	&	\textbf{Pruebas}	&	\textbf{Sobrecarga}	&	\textbf{TOTAL} \\
& \textbf{(horas)}	&	\textbf{(horas)}	&	\textbf{(horas)}	&	\textbf{(horas)}	&	\textbf{(horas)}	&	\textbf{(horas)} \\
\hline
\hline
\endfirsthead
& \textbf{Análisis}	&	\textbf{Diseño}	&	\textbf{Programación}	&	\textbf{Pruebas}	&	\textbf{Sobrecarga}	&	\textbf{TOTAL} \\
& \textbf{(horas)}	&	\textbf{(horas)}	&	\textbf{(horas)}	&	\textbf{(horas)}	&	\textbf{(horas)}	&	\textbf{(horas)} \\
\hline
\hline
\endhead

%FOOT
\hline \multicolumn{7}{r}{\textit{Continúa en la siguiente página}} \\
\endfoot
\endlastfoot

%table
Jefe de Proyecto	&	52,05	&	197,80	&	104,10	&	78,08	&	390,39	&	822,42	\\
Analista	&	364,36	&	416,42	&	0,00	&	0,00	&	35,14	&	815,92	\\
Gestión de la Configuración	&	41,64	&	197,80	&	0,00	&	0,00	&	269,37	&	508,81	\\
Responsable de Calidad	&	41,64	&	197,80	&	104,10	&	156,16	&	39,04	&	538,74	\\
Responsable de Pruebas	&	10,41	&	10,41	&	416,42	&	546,55	&	39,04	&	1.022,82	\\
Desarrollador 1 &	5,21	&	10,41	&	728,73	&	0,00	&	3,90	&	748,25	\\
Desarrollador 2	&	5,21	&	10,41	&	728,73	&	0,00	&	3,90	&	748,25	\\	\hline \hline
\textbf{TOTAL}	&	\textbf{520,52}	&	\textbf{1.041,04}	&	\textbf{2.082,08}	&	\textbf{780,78}	&\textbf{780,78}	&	\textbf{5.205,20}	\\	\hline

\caption{Reparto de horas}\\
\label{tab:repHoras}
\end{longtable}
\end{center}




\subsection{Salarios de los empleados}
\par Tras estudiar en el apartado anterior el número de horas que se estima realizará cada uno de los empleados en el proyecto, y sabiendo también el coste de los mismos por hora, a continuación se expone el salario total que percibirá cada uno de los empleados. Es esta la información que contiene la tabla \ref{tab:costePersonal}.

\begin{table}[H]
\begin{center}
\begin{tabular}{l l c c}
\textbf{CARGO} & \textbf{NOMBRE} & \textbf{TOTAL HORAS} & \textbf{COSTE (\euro)}\\ \hline \hline
Jefe de Proyecto & Alberto García & 822,42 & 28.784,76\\
Analista de Sistemas & Daniel González & 815,92 & 24.477,45\\
Gestión de Configuración & Juan Abascal & 508,81 & 12.720,21\\
Responsable de Calidad & Adriana Lima & 538,74 & 13.468,46\\
Responsable de Pruebas & Carlos Olivares & 1.022,82 & 25.570,77\\
Desarrollador 1 & Carlos Tormos & 748,25 & 14.964,95\\
Desarrollador 2 & Irina Saik & 748,25 & 14.964,95\\ \hline \hline
\textbf{TOTAL} & & & \textbf{134.951,55} \\ \hline
\end{tabular}
\caption{Coste de empleados.}
\label{tab:costePersonal}
\end{center}
\end{table}



\subsection{Equipos informáticos}
\par Para el desarrollo del proyecto haremos uso de los equipos informáticos indicados en la tabla \ref{tab:hardware}. En ella se puede ver el coste total de los dispositivos. Sin embargo, al usarlos únicamente durante los seis meses que dura el proyecto, en la misma se indica el coste que supondrá el uso de los mismos durante ese periodo de tiempo (amortización). Para el cálculo de la misma, se ha supuesto que todos los equipos se amortizan en 3 años.

\begin{table}[H]
\begin{center}
\begin{tabular}{l c c c c }
\textbf{DESCRIPCIÓN} & \textbf{UNIDADES} & \textbf{PRECIO UNITARIO (\euro)} & \textbf{TOTAL (\euro)} & \textbf{AMORTIZACIÓN (\euro)}\\ \hline \hline
Tablets IPAD & 2 & 400 & 800 & 133,28\\
Ordenadores MAC & 2 & 1400 & 2.800 & 466,48\\
Ordenadores HP & 3 & 800 & 2.400 & 399,84\\
Servidor dedicado & 1 & 300\euro/mes & 1.800 & 300\\
Impresora & 1 & 400 & 800 & 133,33\\ \hline \hline
\textbf{TOTAL} & & & \textbf{8.600} & \textbf{1.432,93}\\\hline
\end{tabular}
\caption{Hardware informático.}
\label{tab:hardware}
\end{center}
\end{table}



\subsection{Herramientas del software}
\par Serán necesarias las licencias de los programas indicados en la tabla \ref{tab:software} para el desarrollo del proyecto. Para la parte de desarrollo, se utilizará el control de versiones git mediante el programa BitBucket, y el editor de código Atom. Para el desarrollo de la documentación necesaria se utilizará Office. Para la gestión del proyecto y el control de tareas y tiempos se utilizarán los programas Toggle y Trello.
\par Así mismo, para la gestión general del proyecto y la comunicación entre miembros del equipo se usaran las \textit{suits} Google Apps for Work y Slack.
\par Por otro lado, para la elgaboración de las presentaciones y el material gráfico se utilizará tanto Office como Photoshop respectivamente.


\begin{table}[H]
\begin{center}
\begin{tabular}{l c c c}
\textbf{DESCRIPCIÓN} & \textbf{UNIDADES} & \textbf{PRECIO UNITARIO} & \textbf{TOTAL (\euro)}\\ \hline \hline
Licencias Office365 & 7 & 8,80\euro/mes & 52,8\\
Licencia Toggle & 7 & 9\euro/mes & 54\\
Licencia Trello & 7 & 10\euro/mes & 60\\
Licencia Slack & 7 & 7,5\euro/mes & 45\\
Google Apps for Work & 7 & 4\euro/mes & 24\\
Licencia Photoshop & 3 & 19,99\euro/mes & 119,94\\
Licencia Atom & 7 & 0\euro/mes & 0\\
Licencia BitBucket & 7 & 0\euro/mes & 0\\ \hline \hline
\textcolor{red}{\textbf{TOTAL}} & & & \textcolor{red}{\textbf{355,74}}\\ \hline
\end{tabular}
\caption{Software informático.}
\label{tab:software}
\end{center}
\end{table}



\subsection{Material fungible}
\par Será necesario distinto material de oficina, así como fotocopias y recambios de la impresora, para el desarrollo del proyecto. Pueden verse estos costes en la tabla \ref{tab:fungible}. Se estima que se imprimirán unas dos mil páginas entre los documentos internos, los presentados al cliente y los documentos oficiales requeridos. Sabiendo que el coste del tóner es de 42,95 \euro y estimando una duración de 1200 páginas por tóner, se requerirán dos tóners.
Como material de oficina, se necesitarán los folios usados (un paquete de 2500 tiene un valor de 24,36 \euro), bolígrafos (tanto normales como \textit{veleda}), grapadora con grapas y similar. Se estima el coste de todo ello en 200 \euro.


\begin{table}[H]
\begin{center}
\begin{tabular}{l c}
\textbf{DESCRIPCIÓN} & \textbf{TOTAL (\euro)}\\ \hline \hline
Recambios de impresora & 85,90\\
Material de oficina & 200\\ \hline \hline
\textbf{TOTAL} & \textbf{285,90}\\ \hline
\end{tabular}
\caption{Material fungible.}
\label{tab:fungible}
\end{center}
\end{table}


\subsection{Material de pruebas}
\par Para poder realizar las pruebas del producto desarrollado se necesita el hardware que soporte nuestro software. Así, el equipamiento que se estima será necesario está contenido en la tabla \ref{tab:pruebas}. A priori, serán necesarias tres cámaras, una antena GPS, un sensor de distancia y una Rapsberry y un Arduino como procesadores de la información. Sin embargo, este material podrá variar en función de la solución definitiva escogida más adelante. Estimamos que esta variación será de $\pm20\%$.

\begin{table}[H]
\begin{center}
\begin{tabular}{l c}
\textbf{DESCRIPCIÓN} & \textbf{TOTAL (\euro)}\\ \hline \hline
3 cámaras NetGear & 400,71\\
Antena GPS Garmin GA 25MCX & 19,53\\
Sensor de distacia & 19,53\\
Quit Rapsberry Pi 3 & 85\\
Arduino Mega 2560 & 35\\ \hline \hline
\textbf{TOTAL} & \textbf{559,77}\\ \hline
Desviación 20\% & 119,95\\ \hline  \hline
\textbf{TOTAL CON DESV.} & \textbf{617,72}\\ \hline
\end{tabular}
\caption{Material de pruebas.}
\label{tab:pruebas}
\end{center}
\end{table}


\subsection{Viajes y dietas}
\par A lo largo del proyecto se celebrarán reuniones con los distintos \textit{stakeholders} del proyecto, lo que collevará tanto gastos de la gasolina utilizada en los viajes como de las posibles comidas a las que serán invitados dichos \textit{stakeholders}. Así, se estima que se realizarán unos 5.000 km a lo largo del proyecto. Con un consumo medio de $5,7 litros /100km$ y un coste medio de gasolina de $1,41 \euro/litro$, el coste total de gasolina será de 400 \euro.
\par Por otro lado, se estia que se mantendrán 15 comidas con los \textit{stakeholders} a un precio medio de 200 \euro la comida (seis comensales de media).

\begin{table}[H]
\begin{center}
\begin{tabular}{l c}
\textbf{DESCRIPCIÓN} & \textbf{TOTAL (\euro)}\\ \hline \hline
Gasolina & 400\\
Comidas & 3.000\\ \hline \hline
\textbf{TOTAL} & \textbf{3.400}\\ \hline
\end{tabular}
\caption{Viajes y dietas.}
\label{tab:viajes}
\end{center}
\end{table}


\subsection{Costes indirectos}
En la siguiente tabla mostramos los costes indirectos derivados de las reuniones que mantendrá el equipo y su espacio de trabajo. Al no disponer de oficina física, no existen gastos de electricidad o similares. Sin embargo, si existen gastos asociados al alquiler de una sala co-working, que se alquilará durante 5 horas al día. Así se refleja en la tabla \ref{tab:indirectos}, cuyo cálculo se realiza mediante un coste de 40 euros la hora un total de 5 horas semanales.

\begin{table}[H]
\begin{center}
\begin{tabular}{l c}
\textbf{DESCRIPCIÓN} & \textbf{TOTAL}\\ \hline \hline
Alquiler espacio co-working (Sala Tokio - Impact Hub Madrid) & 1000\euro/semana\\ \hline \hline
\textbf{TOTAL} & \textbf{21.000\euro}\\ \hline
\end{tabular}
\caption{Costes indirectos.}
\label{tab:indirectos}
\end{center}
\end{table}


\section{Costes totales}
\par A continuación, se muestra el presupuesto final del proyecto, desglosando en los distintos costes que lo forman. La duración de dicho proyecto es de 21 semanas. El IVA aplicado es del 21\%. Para el cálculo de este presupuesto se han usado los costes calculados en el Capítulo \ref{chap:costes}.

\begin{table}[H]
\begin{center}
\begin{tabular}{l c}
\textbf{DESCRIPCIÓN} & \textbf{TOTAL}\\ \hline \hline
Sueldo del equipo de trabajo & 134.951,55\\
Amortización de Equipos informáticos & 1.432,93\\
\textcolor{red}{Software informático} & \textcolor{red}{409,74}\\
\textcolor{red}{Material fungible} & \textcolor{red}{400}\\
\textcolor{red}{Material de pruebas} & \textcolor{red}{0,00}\\
Viajes y dietas & 3.400\\
Costes indirectos & 21.000\\ \hline \hline
\textbf{TOTAL} & 161.594,22\\ \hline
\end{tabular}
\caption{Resúmen de costes totales.}
\label{tab:resumenTotal}
\end{center}
\end{table}

En esta tabla se muestra el coste del proyecto sin I.V.A, así como, el riesgo y el beneficio a obtener por la empresa.
\begin{table}[H]
\begin{center}
\begin{tabular}{l c}
\textbf{DESCRIPCIÓN} & \textbf{TOTAL}\\ \hline \hline
Coste del proyecto (sin IVA) &  161.648,39\\
Riesgo (en porcentaje) & 15\% \\
Beneficio (en porcentaje)** & 15\% \\ \hline \hline
\textbf{TOTAL (sin IVA)} & 213.708,35\\ \hline \hline
IVA 21\% & 44.878,76 \\\hline \hline
\textbf{TOTAL} & 258.587,11\\ \hline
\end{tabular}
\caption{Riesgos y beneficios.}
\label{tab:total}
\end{center}
\end{table}

\appendix
\chapter{Estimación por casos de uso}\label{sec:apEst}

Para la estimción del tiempo total que se ha de dedicar al proyecto se ha usado la estimación por casos de uso. De esta forma, lo primero que se ha realizado ha sido identificar los actores de que actuaran en los casos de uso. Estos actores son los objetos (árboles, postes y similares), el vehículo del usuario y el propio conductor, as señales de tráfico y el GPS. Así, y como se puede ver en la tabla \ref{tab:uaw}, se le asigna una dificultad y una ponderación a cada uno de los actores que participan en los casos de uso. Ello refleja el factor de peso de los actores sin ajustar. Sin embargo, no basta con el factor de peso de los actores, sino que se ha de calcular el factor de peso de los casos de uso sin ajustar. En este caso, se le han asignado los factores de ponderación en función del tipo de caso como puede verse en la tabla \ref{tab:ucw}. De esta forma, el total de los puntos de casos de uso sin ajustar sería 130, el equivalente a la suma de los totales de \ref{tab:uaw} y \ref{tab:ucw}.
\par Tras ello, se han calculado los factores de dificultad técnica y los ambientales (véanse las tablas \ref{tab:tcf} y \ref{tab:ef}). Con ello, multiplicando los factores ambientales, los técnicos y el total de puntos de casos de uso sin ajustar, se ha obtenido que el total de puntos de casos de uso ajustados son 104,104. De esta forma, se ha calculado que el total de horas hombre es de 2082,08 en tiempo de programación. Por ello, se ha calculado que el total de horas que se deberán dedicar a este proyecto es de $2082,82/0,4=5205,2$, ya que, como se puede ver en la tabla \ref{tab:porcentajeAct}, el porcentaje que se estima se dedicará a programación será del 40\%.
\par Con esta estimación de horas totales, en la tabla \ref{tab:repHoras}, como ya se ha mencionado, se ha realizado el reparto de horas entre los miembros del equipo de forma que el total de las mismas coincida con el total estimado.

\begin{table}[!hb]
\begin{center}
\begin{tabular}{l l c}
\textbf{ACTOR} & \textbf{DIFICULTAD} & \textbf{PONDERACIÓN}\\ \hline \hline
Objeto externo & Medio & 2  \\
Propio vehículo & Medio & 2\\
Conductor & Complejo & 3\\
Señales & Medio & 2\\
GPS & Fácil & 1\\ \hline
\textbf{TOTAL} &  & \textbf{10}\\ \hline \hline
\end{tabular}
\caption{Factor de peso de los actores sin ajustar.}
\label{tab:uaw}
\end{center}
\end{table}


\begin{table}[!hb]
\begin{center}
\begin{tabular}{l c c c}
\textbf{DIFICULTAD} & \textbf{\#CASO DE USO} & \textbf{PONDERACIÓN} & \textbf{TOTAL}\\ \hline \hline
Fácil & 5 & 5 & 25  \\
Medio & 5 & 10 & 50\\
Complejo & 3 & 15 & 45\\ \hline
\textbf{TOTAL} &  & & \textbf{120}\\ \hline \hline
\end{tabular}
\caption{Factor de peso de los casos de uso sin ajustar.}
\label{tab:ucw}
\end{center}
\end{table}


\begin{table}[!hb]
\begin{center}
\begin{tabular}{l l c c c}
\textbf{TFC} & \textbf{DESCRIPCIÓN} & \textbf{PESO} & \textbf{VALOR} & \textbf{FACTOR}\\ \hline \hline
T1	&	Sistema distribuido	&	2	&	1	&	2	\\
T2	&	Tiempo de respuesta	&	1	&	5	&	5	\\
T3	&	Eficiencia por el usuario	&	1	&	2	&	2	\\
T4	&	Proceso interno complejo	&	1	&	5	&	5	\\
T5	&	Re-usabilidad	&	1	&	2	&	2	\\
T6	&	Facilidad Instalación	&	0,5	&	5	&	2,5	\\
T7	&	Facilidad de uso	&	0,5	&	5	&	2,5	\\
T8	&	Portabilidad	&	2	&	3	&	6	\\
T9	&	Facilidad de cambio	&	1	&	5	&	5	\\
T10	&	Concurrencia	&	1	&	5	&	5	\\
T11	&	Objetivos especiales de seguridad	&	1	&	5	&	5	\\
T12	&	Acceso directo a terceras partes	&	1	&	1	&	1	\\
T13	&	Facilidades especiales de entrenamiento a usuarios finales	&	1	&	1	&	1	\\ \hline
\textbf{TOTAL} & & & & \textbf{120}\\ \hline \hline
\multicolumn{2}{l}{}\textbf{Factores técnicos} & \textbf{1,04} & & \\ \hline \hline
\end{tabular}
\caption{Peso de los factores de complejidad técnica.}
\label{tab:tcf}
\end{center}
\end{table}

\begin{table}[!hb]
\begin{center}
\begin{tabular}{l l c c c}
\textbf{EF} & \textbf{DESCRIPCIÓN} & \textbf{PESO} & \textbf{VALOR} & \textbf{FACTOR}\\ \hline \hline
E1	&	Familiaridad con el modelo del proyecto usado	&	1,5	&	2	&	3	\\
E2	&	Experiencia en la aplicación	&	0,5	&	2	&	1	\\
E3	&	Experiencia OO	&	1	&	4	&	4	\\
E4	&	Capacidad del analista líder	&	0,5	&	4	&	2	\\
E5	&	Motivación	&	1	&	5	&	5	\\
E6	&	Estabilidad de los requerimientos	&	2	&	5	&	10	\\
E7	&	Personal media jornada	&	-1	&	0	&	0	\\
E8	&	Dificultad en lenguaje de programación	&	-1	&	4	&	-4	\\ \hline
\textbf{TOTAL} & & & & \textbf{21}\\ \hline \hline
\multicolumn{2}{l}{}\textbf{Factores ambientales} & \textbf{0,77} & & \\ \hline \hline
\end{tabular}
\caption{Peso de los factores ambientales.}
\label{tab:ef}
\end{center}
\end{table}

\begin{table}[!hb]
\begin{center}
\begin{tabular}{l c c}
\textbf{ACTIVIDAD} & \textbf{PORCENTAJE} & \textbf{REAL}\\ \hline \hline
Análisis & 10\% & 520,52 \\
Diseño & 20\% & 1.041,04 \\
Programación & 40\% & 2.082,08 \\
Pruebas & 15\% & 780,78 \\
Sobrecarga & 15\% & 780,78 \\ \hline
\textbf{TOTAL} & \textbf{100\%} & \textbf{5.205,2}\\ \hline \hline
\end{tabular}
\caption{Peso de los factores ambientales.}
\label{tab:porcentajeAct}
\end{center}
\end{table}
