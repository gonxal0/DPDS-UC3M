\section{Estudio de las alternativas de solución}
El objetivo de este punto es la descripción de tres posibles alternativas que satisfacen los requisitos especificados y, por lo tanto, serán soluciones válidas. El punto de partida en el planteamiento de estas alternativas serán los requisitos y el diagnóstico de la situación actual del entorno con el que el sistema interactuará en el futuro. Los subsistemas que se han identificado previamente servirán como apoyo para definir los componentes hardware necesarios y las alternativas software que se ofrecerán al cliente. Todas las alternativas de cada subsistema deberán contemplar los siguientes puntos:

\begin{itemize}[-]
\item \textbf{Control de punto ciego:} es el subsistema encargado de notificar al conductor en caso de que haya un objeto en un área delimitada, llamada punto ciego [\hyperref[tab:RF-01]{RF-01}, \hyperref[tab:RF-02]{RF-02}]. Será implementado mediante un hardware colocado en los laterales exteriores del vehículo. A su vez, será necesario un software que analice los datos recibidos del hardware y determine si existe un objeto potencialmente peligroso, en cuyo caso, se notificará al conductor mediante los retrovisores [\hyperref[tab:RI-01]{RI-01], indicando en qué lado se encuentra el objeto. En caso de que esté activada la notificación y el conductor active el intermitente hacia ese lado, se generará una notificación sonora [\hyperref[tab:RI-02]{RI-02}]. En la actualidad ya se encuentran desarrollados y se ofrecen como un paquete opcional al comprar muchos de los vehículos del mercado.
Aviso de cambio de carril: esta funcionalidad debe de notificar al conductor cuando el vehículo comience a salirse del carril por el que va circulando. Sin embargo, el sistema de notificación será desactivado si el intermitente está accionado [\hyperref[tab:RF-08]{RF-08}]. Para poder implementarlo será necesario un hardware situado en la zona frontal del vehículo, que envíe al sistema información acerca de la trayectoria que sigue el carril [\hyperref[tab:RF-05]{RF-05].  El sistema comparará esta trayectoria con una proyección de la trayectoria que va a seguir el vehículo en función del ángulo del volante y la velocidad [\hyperref[tab:RF-06]{RF-06}]. Si el sistema detecta una desviación entre ambas trayectorias superior al 10\%, hará vibrar al volante [\hyperref[tab:RF-07]{RF-07}, \hyperref[tab:RI-03]{RI-03}]. En el momento en que esta desviación supere el 15\%, el sistema corregirá un máximo de 5 grados la dirección del volante [\hyperref[tab:RF-09]{RF-09}].
\item \textbf{Alerta de velocidad:}  este subsistema debe notificar al conductor a través del panel de abordo cuando se supere la velocidad máxima permitida para ese tipo de vía [\hyperref[tab:RI-04]{RI-04}]. Para obtener este dato, en primer lugar deberá recibir imágenes a través de un hardware colocado en el exterior del vehículo y reconocer en ellas señales de tráfico [\hyperref[tab:RF-10]{RF-10}]. Una vez reconocidas, las clasificará por formas y colores [\hyperref[tab:RF-11]{RF-11] y reconocerá las señales de velocidad [\hyperref[tab:RF-12]{RF-12]. El sistema determinará de esta forma la velocidad máxima para ese tipo de vía [\hyperref[tab:RF-13]{RF-13}], y calculará también la velocidad mínima por debajo de la cual se dificulta la circulación [RF-14]. Paralelamente, el sistema utilizará la posición GPS para detectar el tipo de vía en la que se encuentra el vehículo [\hyperref[tab:RF-15]{RF-15}] y tendrá almacenados los mapas de forma local [\hyperref[tab:RI-10]{RI-10}], y las velocidades máximas establecidas para cada tipo de vía en el código de circulación [\hyperref[tab:RF-16]{RF-16}]. Por defecto, cuando el sistema detecte que se ha producido un cambio de vía, se utilizará la velocidad máxima establecida en el código de circulación [\hyperref[tab:RF-17]{RF-17}]. Cuando el sistema detecte una señal de velocidad, se deberá actualizar la velocidad máxima permitida [\hyperref[tab:RF-18]{RF-18}], dando prioridad a las señales luminosas [\hyperref[tab:RF-19]{RF-19}] y a las señales de obra [\hyperref[tab:RF-20]{RF-20}]. Si el sistema no detecta ninguna señal de velocidad en un tiempo superior a 5 minutos, se establecerá la velocidad máxima del código de circulación para ese tipo de vía [\hyperref[tab:RF-21]{RF-21}]. El conductor podrá configurar el coche para que este nunca supere la velocidad máxima [\hyperref[tab:RF-22]{RF-22}], así como desactivar las alertas [\hyperref[tab:RI-04]{RI-04}, \hyperref[tab:RF-23]{RF-23}]. Las alertas se volverán a activar al reiniciar el sistema [\hyperref[tab:RF-24]{RF-24}].
\item \textbf{Sistema de alerta por pérdida de atención:} este subsistema deberá detectar cuando el conductor se ha quedado dormido [\hyperref[tab:RF-25]{RF-25}], producir una alerta para que reaccione [\hyperref[tab:RI-05]{RI-05}] y detener el vehículo si este no reacciona en menos de 3 segundos, activando las luces de emergencia [\hyperref[tab:RF-28]{RF-28}]. Paralelamente, también detectará los errores en la forma de conducir, analizando el comportamiento del conductor [\hyperref[tab:RF-26]{RF-26}] y recomendando a este un descanso si se detecta pérdida de atención por fatiga al volante [\hyperref[tab:RF-27]{RF-27}].
\item \textbf{Llamada de emergencia:} en caso de colisión, el sistema deberá enviar un mensaje [\hyperref[tab:RF-30]{RF-30}] indicando la posición del vehículo, la velocidad a la que se ha producido el impacto y el número de ocupantes del vehículo [\hyperref[tab:RF-32]{RF-32}]. Acto seguido, se activarán todas las luces de emergencia [\hyperref[tab:RF-33]{RF-33}]. La llamada de emergencia seguirá el protocolo recogido en la norma europea EN 16072 [\hyperref[tab:RI-09]{RI-09].
\item \textbf{Alerta de precolisión:} este subsistema deberá detectar obstáculos en la trayectoria del vehículo [\hyperref[tab:RF-34]{RF-34}] y analizar el riesgo de que se vaya a producir una colisión [\hyperref[tab:RF-35]{RF-35}]. En caso de que este riesgo sea superior al 50\%, se notificará al conductor de una posible colisión [\hyperref[tab:RF-36]{RF-36}, \hyperref[tab:RF-11]{RF-11}], esperando a que el conductor pise el freno para frenar lo necesario como para evitar la colisión [\hyperref[tab:RF-37]{RF-37]. SI esta probabilidad es superior al 70\%, se considerará como una colisión inminente [\hyperref[tab:RF-38]{RF-38}], y el sistema reducirá la velocidad [\hyperref[tab:RF-39]{RF-39}], ajustará los cinturones de seguridad [\hyperref[tab:RF-40]{RF-40}], optimizará el rendimiento de los airbag [\hyperref[tab:RF-41]{RF-41}] y cerrará las ventanillas [\hyperref[tab:RF-42]{RF-42}]. Si se evita el accidente, el sistema revertirá todas las acciones [\hyperref[tab:RF-43]{RF-43}].
\end{itemize}

A continuación, presentaremos alternativas para cada uno de los subsistemas, generando en su conjunto tres alternativas distintas para el sistema global. No obstante, los subsistemas de llamada de emergencia y de pérdida de atención tendrán una única alternativa, que será explicada a continuación, y en la cual se expondrá porque no se propondrán al cliente otras alternativas. Por lo tanto, cualquiera de las alternativas globales estará completada por la única alternativa posible de cada uno de estos subsistemas.

Por otro lado, el subsistema de alerta de velocidad dispondrá únicamente de dos alternativas, debido a su complejidad de funcionamiento. De esta forma, las alternativas globales de funcionamiento podrán ser completadas con cualquiera de las dos alternativas del mencionado subsistema.
