\subsection{Alternativa I}

\par Para esta primera alternativa se le ha propuesto al cliente que sistema central esté controlado por una Raspberry Pi 3B. Dado que las Raspberry no vienen con sistema operativo instalado, se ha considerado la opción de instalar Raspbian Jessie with Pixel para poder procesar toda la información.
\par Para poder procesar las imágenes se utilizará un System-on-Chip con FPGA. Con un solo FPGA controlaremos el punto ciego, las alertas de velocidad, el sistema de precolisión y el cambio de carril, ya que es un sistema muy potente que es capaz de procesar las imágenes recibidas de los cuatro subsistemas. Las placas FPGA compartirán los datos con la Raspberry mediante un puerto Ethernet, el cual se rige bajo el estándar IEEE 802.3an.
\par Los mapas del país vendrán instalados ya en el sistema operativo, gracias a la librería compatible con Raspbian llamada Tangram.

\par A continuación se detallan las alternativas propuestas para cada uno de los subsistemas:

\begin{itemize}[-]
    \item Para el \textbf{control del punto ciego} se utilizarán dos cámaras de escaneo por área para determinar si es conveniente realizar la maniobra. Dichas cámaras tendrán una lente de 3.5mm para aumentar la precisión. Las imágenes captadas por las cámaras serán procesadas por un FPGA. Cuando sea necesario, se mandará notificaciones a la Raspberry poder comunicarse con el usuario.
    \item En los subsistemas \textbf{cambio de carril y alerta de velocidad} se utilizará otra cámara de escaneo por área (mismo modelo que en el control de punto ciego), y la misma FPGA para procesar las imágenes.
\end{itemize}

\begin{table}[H]
\begin{center}
\begin{tabular}{p{5cm} p{8cm}}
\multicolumn{2}{c}{\textbf{Especificación de la Alternativa I} } \\
\hline \hline

\textbf{Sistema General} &
Raspberry Pi 3B
\begin{itemize}
    \item Precio: 39,90 \euro
    \item RAM: 1GB
    \item Procesador: Chipset Broadcom BCM 2387 1,2GHz de cuatro núcleos ARM Cortex-A53
    \item Conexión: Ethernet Socket 10/100 BaseT
\end{itemize}
Sistema Operativo Raspbian Jessie with Pixel
\begin{itemize}
    \item Version: Enero 2017
    \item Fecha de lanzamiento: 11/01/2017
\end{itemize}
\\ \hline

\textbf{Control del punto ciego} &
2 Cámaras de escaneo por área
\begin{itemize}
    \item Precio: 560,00 \euro
    \item Modelo: Basler ace USB3 acA640-90uc
    \item Color: Sí
    \item FPS: 90
    \item Resolución: 659x490
    \item Puerto: USB3
\end{itemize}
Lentes
\begin{itemize}
    \item Precio: 575,00 \euro
    \item Modelo: Compact Fixed Focal Length 3,5 mm
\end{itemize}
Cable USB3
\begin{itemize}
    \item Precio: 48,00 \euro
\end{itemize}
FPGA
\begin{itemize}
    \item Precio: 499,20 \euro
    \item Modelo: Zynq®-7000 XC7Z030-L2FFG676I-ND
    \item Celdas lóicas: 125000
\end{itemize}


\\ \hline

\textbf{Cambio de carril, Alerta pre-colisión y Alerta de velocidad} &
Cámara de escaneo por área
\begin{itemize}
    \item Precio: 560,00 \euro
    \item Modelo: Basler ace USB3 acA640-90uc
    \item Color: Sí
    \item FPS: 90
    \item Resolución: 659x490
    \item Puerto: USB3
\end{itemize}
Lentes
\begin{itemize}
    \item Precio: 575,00 \euro
    \item Modelo: Compact Fixed Focal Length 3,5 mm
\end{itemize}
Cable USB3
\begin{itemize}
    \item Precio: 48,00 \euro
\end{itemize}
FPGA
\begin{itemize}
    \item Precio: 499,20 \euro
    \item Modelo: Zynq®-7000 XC7Z030-L2FFG676I-ND
    \item Celdas lóicas: 125000
\end{itemize}




\end{tabular}
\caption{Especificación de la Alternativa I}
\label{tab:altI}
\end{center}
\end{table}
